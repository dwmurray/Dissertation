\documentclass[../dissertation.tex]{subfiles}
%\setlength{\epigraphwidth}{.7\textwidth}

\begin{document}

\chapter{Introduction}
\label{ch:intro}

\singlespace
\epigraph{``\emph{Everything starts somewhere, though many physicists disagree. But people have always been dimly aware of the problem with the start of things. They wonder how the snowplough driver gets to work, or how the makers of dictionaries look up the spelling of words}"}{--- \textit{Hogfather by Terry Pratchett}}

\dblspace

%Start close to home: our Sun provides all the energy that we have/use on Earth.
Stars are arguably the fundamental building blocks of the Universe, 
and play a crucial role in the existance of life on Earth. Newborn stars consist of $\sim 71 \%$ Hydrogen, $\sim 27 \%$ Helium, and a smattering of heavier elements. Almost all of the heavier elements, including those that make up the planets of our solar system, for example oxygen, silicon, and iron, as well as elements found in living organisms, like carbon and nitrogen, were created by stars. 

%\subsection{The energy that drives life on Earth}
The energy that fuels life and our modern society comes exclusively from stars, with all but one form coming from our Sun. 
The oil that we burn, the wind energy we capture via turbines, 
and the radiation we capture via solar panels, all have an origin in the nuclear fusion reactions in the center of our Sun. 
The one exception is the fuel used in nuclear reactors. Those elements, including Uranium,  were created in stars that died before our Sun was born.

%\subsection{What makes up a star?}




%\section{How do we observe stars}




Stars have been studied since antiquity, partly because the position of the sun relative to the stars allowed people to predict the timing of the seasons. 
Anaxagoras and Democritus argued that the Milky Way consisted of many stars blurred together, and that the sun was a star. 
While this argument proved to be correct, astronomers long applied it too broadly, asserting that 
there was no true nebulosity, i.e., no gas or other absorbing material between the stars \citep{1997AIPC..393...15T}. 
The idea that there is no gas outside of stars naturally goes hand in hand with the idea that stars are not forming at the present time, 
an idea consistent with the notion that the stars are eternal, e.g., Thomas Carlyle's quote that ``The eternal stars shine out again, as soon as it is dark enough''. 
Trimble argues that the idea that most, if not all, stars formed long ago was held by most astronomers until the 20th century. 

This paradigm changed dramatically in 1930, when Trumpler \citep{1930LicOB..14..154T} showed convincingly that stellar radiation was absorbed by some medium between the stars. 
He did so using two methods, one based on the apparent diameters of open clusters, 
and a second based on the anomalously red colors of stars of a given spectroscopic type known to lie at large distances. 
In the same paper Trumpler noted that the obscuring material was confined to the plane of the Milky Way. 
The amount of gas associated with this obscuring material was not known, but was assumed to be small compared to the stellar mass of the Galaxy.

It took another 20 years, and the discovery of 21 cm emission from neutral hydrogen, to establish the presence of a 
large mass of gas in the Milky Way  \citep{1951Natur.168..356E,1951Natur.168..357M,1951Natur.168..358P}.

While the existence of a large mass of molecular gas was suspected by some, 
its presence was not widely acknowledged until the discovery by \citet{1970ApJ...161L..43W}
of emission from carbon monoxide (CO) molecules. 
It was rapidly realized that the molecular gas, which had a total mass of about $1.6\times10^9M_\odot$ \citep{1993AIPC..278..267D,2017ApJ...834...57M} 
was confined mostly inside the solar circle, i.e., within about 8 kiloparsecs (kpc) of the Galactic center, 
and that it occupied a small fraction of the volume occupied by the atomic gas. 
The total mass of the latter is about $7\times10^9M_\odot$. 
The atomic gas extends much further out than the molecules, to 20 kpc in the Milky Way. 

By about this same time (the middle of the twentieth century), most astronomers had arrived at the view that star formation was an ongoing process, 
driven primarily by the work of \citet{1939PhRv...55..434B}
and others showing that starlight was powered by nuclear fusion. 
The high luminosities of O and B stars convinced Bethe that such stars had to have formed recently (by astronomical standards), less than a few hundred million years ago. 
Our current understanding of stellar evolution indicates that stars of 40 or more solar masses live only 4 million years. 
Since we observe about 10,000 such stars in the Milky Way, the current estimate is that about two solar masses worth of stars are born every year \citep{2011AJ....142..197C}. 

%The recognition that stars consisted primarily of hydrogen and helium was gradual, starting with Payne (1925) PNAS vol. 11 p 192.
\section{The Phenomenology of the Interstellar Medium (ISM)}
The ISM provides the constituents from which stars form, including both molecules of gas, and dust grains. 
The total mass within $R \le 60 \kpc$ of the Galactic center is $M = 4.6 \times 10^{11} \Msun$. 
Baryons provide $M = 9.5 \times 10^{10} \Msun$ with a gas fraction of 13\%. 
The HI mass is $M = 8 \times 10^9 \Msun$, the warm ionized medium contributes $M = 2 \times 10^9 \Msun$, and molecular gas is $M = 2.5 \times 10^9 \Msun$ \citep[p. 35]{2009ARA&A..47...27K}. The hot ionized gas, with temperatures of order $10^6\,{\rm K}$ or higher, contains little mass, but occupies most of the volume for heights more than one kpc above the disk mid-plane (near the mid-plane, it fills only about 20\% of the volume).

\subsection{Warm Ionized Gas}
Regions of warm ionized gas where the gas temperature  exceeds several thousand Kelvin \citep[p. 47]{2009ARA&A..47...27K}, are common in the inner disk, and are often referred to as $\HII$ regions. Warm ionized gas is also seen above and below the neutral gas in the outer disk.
Observers see $\HII$ regions surrounding massive O stars. 
The radiation given off by these ``rock stars'' of the universe is enough to dissociate any molecules, and even strip electrons off protons. These ionized electrons then share their energy (the difference between the energy of the photon that ionized the host atom and the ionization potential of that host) with the surrounding gas, heating it.
I will discuss the radiation from massive stars in Section \ref{sec:Intro_FF_and_CO}. The ionized layers overlying the neutral disk at large radii are heated by the cosmic ultraviolet background radiation.

\subsection{Neutral Gas}
Neutral hydrogen, which is traced observationally by its 21-cm line emission (short radio wave), exists in a variety of environments. 
The densities range from $1 \,  - \, 10^{-4} \, {\rm particles \, cm}^{-3}$.
Temperatures range from $10 \, {\rm K}$ to temperatures as high as $10^3 \sim 10^4 \, {\rm K}$. 
As Peter Kalberla and J{\"u}rgen Kerp succicintly state: 
``Studying neutral atomic hydrogen means first focusing on the thermal pressure.'' \citep[p. 47]{2009ARA&A..47...27K}
Observers typically differentiate the neutral gas by temperature, referring to ``warm'' and ``cold'' neutral gas. While it is often assumed that the gas is in both thermal and pressure equilibrium, recent work suggests a more dynamic picture, with nearly half the neutral gas in a thermally unstable region \citep{2003ApJ...586.1067H}. 
%``warm'' and ``cold'' neutral gas.

The width of the 21-cm emission line varies dramatically from sight line to sight line. Along sight lines with the narrowest lines, the width is consistent with thermal broadening. However, in many directions, it is clear that the emitting gas is undergoing large scale high velocity flows, producing linewidths in excess of $30\kms$.

\subsubsection{Warm Neutral Medium (WNM)} %Neutral atomic hydrogen 
There are a multitude of processes that contribute to the ``high'' temperature of the WNM in the ISM; 
for example, heating from hydrodynamic and magnetohydrodynamic processes, heating due to interstellar shocks, heating due to soft X-rays and stellar radiation, longward of the Lyman limit.
A key take away from this list is that heating in the ISM is dependent upon location within the ISM. 
In addition, the efficiency of heating depends upon the phase of the material: 
different heating and cooling mechanisms determine the temperature of the gas, depending upon the current temperature, density and ionization of the medium \citep[p. 48]{2009ARA&A..47...27K}.
The velocity dispersion in the WNM is $\Delta v \simeq 24 \kms$ \citep[p. 49]{2009ARA&A..47...27K}, corresponding to the line of sight velocity dispersion of $\sigma_{\rm los} \approx 10 \kms$. This is comparable to the thermal line width.


\subsubsection{Cold Neutral Medium (CNM)}
In contrast to the WNM, the velocity dispersion in the CNM is $\Delta v \simeq 4 \kms$ \citep[p. 49]{2009ARA&A..47...27K}.
The CNM is heated by a variety of sources, including photoelectric heating by dust grains, by soft X-rays, and hydrodynamic and magnetohydrodynamic effects.
Observers have been constrained primarily to fine structure lines, which are visible in the far infrared regime because the of the low volume density of the CNM. 
In these regions neutral hydrogen is no longer the dominant cooling mechanism: 
``Owing to its low ionization potential below 13.6 eV and its high abundance, [atomic] carbon dominates the cooling of neutral gas traced by the HI 21-cm line'' \citep[p. 48]{2009ARA&A..47...27K}. 
In addition, neutral oxygen plays a signifcant role in cooling neutral gas within the Galactic plane. 
The dispersion or line width seen in the CNM can often be much wider than can be explained by thermal motions. 
In other words, the gas exhibits bulk flows that are supersonic. 


\subsection{Molecular Gas} \label{subsec:intro_molecular_gas}
As the CNM approaches the lower temperatures listed above, molecules such as H$_2$, CO and NH$_3$ can begin to form \citep[p. 47]{2009ARA&A..47...27K}. 
While H$_2$ is the most abundant molecule in the Milky Way, and indeed in the Universe, it is very difficult to detect. 
Since it is symmetric, it has no dipole moment, and as a result it does not radiate efficiently. 
However, molecular hydrogen is often accompanied by the second most abundant molecule, CO. 
In order for either molecule to exist in appreciable abundances, the molecules must not be exposed to large fluxes of Lyman-Werner photons (with energies between $11.2$ and $13.6\,{\rm eV}$), which are capable of dissociating the molecules. Photons of slightly lower energies can dissociate CO molecules, so the latter are usually found slightly deeper into molecular clouds, inside a protective molecular hydrogen envelope.

Molecular gas accounts for only 10\% of the total gas mass in the entire Milky Way Galaxy.
However, much of the molecular gas is inside the Solar Circle (inside the radius from the Galactic Center to our Sun, about 8 - 8.5 kpc). 
Molecular gas accounts for half or more of the gas mass inside the Solar Circle. 


\subsubsection{Giant Molecular Clouds}
Soon after the discovery of interstellar CO by \citet{1970ApJ...161L..43W}, it was realized that most of this emission originated from discrete regions, both on the sky, and in velocity space. These regions came to be called clouds. Much later,
\citet{1987ApJ...319..730S}  showed that the bulk of this gas is in large clouds, with masses upwards of $\approx10^6M_\odot$. 
These massive clouds are known as giant molecular clouds (GMCs).

\citet{1981MNRAS.194..809L} examined early catalogs of GMCs and extracted three laws, now known as Larson's laws. It was later realized that only two of the laws were independent---the third could be derived from the other two. The most famous law relates the size of a GMC and the linewidth of that cloud:
%
\be
\left({\sigma\over \kms}\right) = 1.10\left({L\over \pc}\right)^{0.38}, 
\ee
%
where $\sigma$ is the dispersion velocity, and L is the charateristic size of the cloud.
Larson found a similar relation for the interiors of many of his clouds, i.e., the clouds were composed of substructures that had similar properties to those found on the largest (cloud) scale.
More recent work finds a somewhat larger exponent relating the linewidth to the scale of the observed region, with values clustering around $0.5$.

Larson interpreted this result as evidence that the interstellar medium was turbulent, and that the linewith reflected a turbulent cascade. 
He connected this with the Kolmogorov scaling for sub-sonic turbulence
%
\be
\sigma\sim L^{1/3}.
\ee
%

On large scales the linewidth exceeds the thermal linewidth, indicating that the motion is supersonic, a point Larson appreciated and commented on. Typical Mach numbers for large GMCs are ${\cal M}\approx 10$ in the Milky Way, with much larger values seen in more rapidly star forming galaxies.

Simulations of supersonic turbulence find a relation of the form $\sigma\sim L^{1/2}$,
where $\sigma$ is the dispersion velocity, and L is the length scale on which the velocity dispersion is measured. 
This again suggests that the observed linewidths reflect supersonic motions in the ISM.

Larson also showed that the linewidth scaled with the mass of the GMC
%
\be
\left({\sigma\over \kms}\right) = 0.42\left({M\over M_\odot}\right)^{0.20}.
\ee
%

The size-linewidth relation, with a slightly larger exponent ($\sim0.5$) has been recovered by dozens of groups over the last thirty years, e.g., \citet{1987ApJ...319..730S,2008ApJ...679.1338R,2017ApJ...834...57M}.

\subsubsection{Clumps and cores}
As Larson noted, GMCs are not the only structures identified in the ISM. \citet{2000prpl.conf...97W}  argue that the ISM has structure on all scales, as would be expected if turbulence dominated the kinematics of the gas. Despite the connection they make between the observed properties of the ISM and turbulence, \citet{2000prpl.conf...97W}  do pick out two scales other than that of GMCs, which they call clumps and cores. This nomenclature has infiltrated the literature.

They define clumps on a purely observational basis: ``Clumps are coherent regions in l-−b-−v space, generally identified from spectral line maps of molecular emission.'' The quantities l and b refer to Galactic longitude and latitude, while 'v' refers to the observed velocity of the line emission.

This is in contrast to their definition of cores, which makes reference to a physical property, namely being gravitationally bound: ``Cores are regions out of which single stars (or multiple systems such as binaries) form and are necessarily gravitationally bound.''

\subsection{Deviations from Larson's laws in massive star forming regions}

A surprising and suggesting result was obtained by \citet{1995ApJ...446..665C},%Caselli & Myers
who measured linewidths in massive cloud cores, in which massive stars were forming. In contrast to measurements on larger scales, and on lower mass cores, the massive cores had a modified size-linewidth relation,
%
\be
\left({\sigma\over \kms}\right) = 0.72\left({R\over \pc}\right)^{0.21},
\ee
%
to be compared with their result for low mass star forming cores
%
\be
\left({\sigma\over \kms}\right) = 0.64\left({R\over \pc}\right)^{0.52}.
\ee
%


Following the results of \citet{1995ApJ...446..665C}, 
\citet{1997ApJ...476..730P} observed 150 massive star forming regions, using a 30m telescope. %IRAM 
They did not even see a statistically relevent line width-size relation; the line width was nearly independent of the size of the core they observed.
However, they did find that the line width increases with density; since the density was seen to increase with decreasing radius (the usual result), the implication is that the turbulent line width increased with {\em decreasing} radius, the opposite of Larson's law.



\section{Where Are Stars Born?}
How do observers know where stars form?
Before the 1970's there were no efficient far-infrared detectors. 
Until recently it was impossible to find individual proto-stars, because the regions where they form are optically thick: the visible light from proto-stars is severely diminished.
Since observers couldn't look for young stars directly, they used proxies or tracers instead. 
The two most commonly used tracers are free-free emission, and far infrared emission.

\subsection{Star formation tracers}

Free-free emission arises from ionized gas. When radiation of sufficient energy hits molecular or atomic gas, it dissociates molecules and ionizes atoms, e.g.,  it strips electrons from hydrogen atoms, leaving protons.
The liberated electron then shares its kinetic energy (the difference between the energy of the photon and 13.6 eV, in the case of a hydrogen atom) with the surrounding gas. 
As the electron loses energy, it heats the gas, and emits free-free radiation.
The free-free emission is created as the electrons encounter protons and are decelerated, leading to the other name for such emission, bremsstrahlung (``braking radiation''). 
The associated gas temperatures are around 7,000-10,000 Kelvin. 

Free-free emission is associated with recombination radiation, which is emitted when electrons are recaptured onto hydrogen or other ions. Recombination produces
H$_\alpha$ and H$_\beta$ line emission in the optical band, as well as H$_{109}$ and similar radio recombination lines. 
Emission from other ions, such as [OII] and [OIII] (the square brackets denote `forbidden', or non-dipole, atomic transitions), is also produced. Regions that produce emission from ionized gas are called HII regions. In the Milky Way it is difficult to detect the optical recombination lines, since ISM dust readily absorbs such radiation. However, the optical emission lines are widely used as a star formation tracer in external galaxies. The radio recombination lines are not affected by extinction, but they are very weak, particularly in comparison to the free-free emission. 

The most common source of ionization in the Galaxy is radiation from massive ($\sim 40M_\odot$) stars, which, as we saw above, are less then about 4 Myrs old. Hence free-free emission is a tracer of star formation.

The fact that optical and UV emission is absorbed by dust results in the other most widely used star formation tracer, far infrared (FIR) emission. Starlight (mostly from massive stars, since they dominate the luminosity of a cluster of young stars) heats dust grains, which re-radiate the energy as thermal radiation, typically with temperatures of tens of Kelven. 

More recently, it has become possible to use near- or mid-infrared space-based telescopes to identify proto-stars via their infrared light excess. 
This occurs when the light from a proto-star hits the surrounding accretion disk. Dust in the accretion disk absorbs the stellar radiation, and then re-emits it in the infrared and far infrared. Thus proto-stars appear far brighter in the near or mid-IR than do more evolved stars of the same mass or luminosity. 

\subsection{Correlation between Free-Free \& CO and between FIR \& CO} \label{sec:Intro_FF_and_CO}

Observations of the galactic plane looking at free-free emission and CO emission show that the two tracers are well correlated on large ($\gtrsim 10 \pc$) scales; the same is true of free-free and FIR emission \citep{1989ApJ...339..149S,1988ApJ...334L..51M}. In particular, HII regions are embedded in much larger CO emitting regions. Images suggest that the hot stars appear to have dissociated or expelled the molecular material in the immediate vicinity of the star cluster that produces the ionizing radiation.
Subsequent work by a very large community of astronomers, too numerous to cite, has confirmed that star formation takes place in the densest clumps of GMCs.


\section{Star formation}
Having established that star formation takes place in GMCs, I now very briefly sketch the current large scale picture of star formation.
GMCs in the Milky Way have masses ranging from about $10^4$ to $10^7M_\odot$ (ten thousand to ten million solar masses), 
and are observed to form anywhere from hundreds up to $50,000\,M_\odot$ of stars, the latter exemplified by Westerlund 1 \citep{2017A&A...602A..22A}. 

Astronomers believe (and I will show below) that portions of these
GMCs collapse, initially due to turbulent motions, and later under
their own gravity.  The central parts of the collapsing regions
eventually become dense enough that radiation cannot escape; further
collapse simply heats the gas.  Eventually, the central temperature reaches
several to ten million degrees, initiating thermonuclear reactions.
At this point, a nascent star is born.

Observations tell us that most of these young stars are surrounded by flattened gas discs. 
To paraphrase author Terry Pratchett, this is truly ``the start of things'' astronomical. 
Recent studies of {\it Kepler} planetary systems have shown that at least 1 out of 3 stars have planets \citep{2018arXiv180209526Z}; 
therefore, at least 1 out of 3 discs contains planets. 
%There is observational evidence that planets form in these discs, for example, the famous ALMA image of HL Tau \citep{2015ApJ...808L...3A}.


\subsection{Star Formation Efficiency (SFE)}
Stars are believed to be responsible for regulating the amount of gas in galaxies--by driving winds out of galactic disks-- as well as for the ejection of heavy elements into the intergalactic
medium. In doing so, stars regulate the total fraction of gas that ends up in stars and planets. 
 
Observations of galaxies have shown that, at most, only one quarter of the gas available to a given galaxy is converted into stars, e.g., \citet{2010ApJ...708L..14M}; for most galaxies, the fraction is much smaller. 
Why this should be, and how the fraction is determined, is currently one of the most important topics in astrophysics.

One measure of how effective nature is at converting molecular gas into stars is the star formation effeciency (SFE), denoted by $\epsilon$:
%
\be
\epsilon = \frac{M_*}{M_{\rm gas}},
\ee
%
where $M_*$ is the total mass in stars and $M_{\rm gas}$ is the total mass in gas, in a given region.

A second measure of efficiency is how much gas turns into stars in a free-fall time. If we imagine an isolated cloud of gas with constant density, $\rho(r) \rightarrow \rho$, we can derive the amount of time required for the cloud of gas to collapse to a single point.
This time is the free-fall time:
%
\be
t_{\rm ff} = \sqrt{\frac{3 \pi}{32 G \rho}},
\ee
%
where $ {\rm G} = 6.67 \times 10^{-8}\, {\rm g}^{-1}\cm^3\s^{-2} $ is Newton's constant. At this point it will be useful to introduce the dynamical time and the virial parameter of a gas cloud. The dynamical time is given by
%
\be
\tau_{\rm dyn}\equiv {v\over R},
\ee
%
where $R$ is the size of the cloud (often taken as the squre root of the area of the cloud divided by $\pi$) and $v$ is the turbulent velocity (or the sound speed if the motions are subsonic). 
The dynamical time is essentially the time required for information to spread through the system. 
The virial parameter is
%
\be
\alpha_{\rm vir}\equiv 5{v^2 R\over GM},
\ee
%
where $M$ is the mass of the cloud and v is the line of sight velocity. 
The virial parameter is the ratio of two times the total kinetic energy over the potential energy. 
If the virial parameter is of order unity, then the system is gravitationally bound.

Note that for a uniform density cloud,
%
\be
\alpha_{\rm vir}={15\over 4\pi} \left({t_{\rm ff}\over \tau_{\rm dyn}}\right)^2
\ee
%

In the simple model where gravity is the only force acting, if all of the gas collapses to a point in a free-fall time, then the SFE would 100\%.
The question we need to answer, is does this zeroth-order model hold observationally, and if not, what physics dominates the dynamics?

\section{Star Formation on the galactic scale, or, the Kennicutt-Schmidt law}
In fact, the star formation time on galactic scales is long when compared to the dynamical time. 
\citet{1998ApJ...498..541K} expressed this in the form
%
\begin{equation}\label{eq:intro_KS_law}
\dot{\Sigma}_* = \eta\Sigma_g \tau_{\rm DYN}^{-1}
\end{equation}
%
where $\dot{\Sigma}_*$ is the star formation rate per unit area, $\Sigma_g$ is the 
gas surface density, $\tau_{\rm DYN}$ is the local dynamical time, and $\eta = 0.017$ 
is the efficiency factor.  
The gas surface density is used in this expression (as opposed to the gas volume density) as the surface density is a value which can be found observationally. 
In addition, the gas surface density also appears in the denominator of the Toomre Q parameter which is used to determine the graviational stablity of protostellar accretion disks. 


In our rather naive model, if the gas self-gravity dominates the dynamics, $\eta \sim 1$, so the low efficiency of star formation is surprising. 
More recent work has refined this and similar relations in regard to its dependence on molecular gas \citep{2008AJ....136.2846B} and by taking into account the error distributions of both $\dot\Sigma_*$ and $\Sigma_g$ \citep{2013MNRAS.430..288S}, but the best current estimates of the efficiency of star formation on galactic scales remains low. 

\section{Star Formation on the GMC scale}
Whether this low efficiency applies to scales comparable to giant molecular clouds, with radii of
order $100\pc$, is debated in the literature. The analog of equation (\ref{eq:intro_KS_law}) for a GMC or clump is
%
\be
\dot \Sigma_* = \epsilon_{\rm ff}\Sigma_g t^{-1}_{\rm ff}.
\ee
%
\citet{2010ApJ...723.1019H}, \citet{2010ApJ...724..687L}, \citet{2010ApJS..188..313W}, and \citet{2011ApJ...729..133M} 
find efficiencies $\epsilon_{\rm ff}\approx 0.1$ or higher,  while \citet{2007ApJ...654..304K} and \citet{2012ApJ...745...69K} 
find $\epsilon_{\rm ff}\approx0.01$.
On these small scales, observations also suggest that the efficiency is not universal, but instead 
varies over two to three orders of magnitude, e.g., 
\citealt{1988ApJ...334L..51M,1989ApJ...339..149S,2016ApJ...833..229L}).%2015ApJ...804...18A}).  

There are a number of explanations for the low star formation rate, on 
either small or large scales (although they may not be necessary for the former!). 
These include turbulent pressure support \citep{1992ApJ...396..631M}, support from magnetic fields
\citep{1966MNRAS.132..359S,1976ApJ...207..141M}, and stellar feedback (e.g.
\citealt{1986ApJ...303...39D}).  

%% A number of explanations for this low star formation rate and scatter in the efficiency, on either galactic or GMC scales, have been put forth.
%% On large scales, the leading candidate is stellar feedback, e.g. \citet{1986ApJ...303...39D}, in which supernovae limit the amount of dense gas. 
%% On small scales, these include turbulent pressure support \citep{1992ApJ...396..631M} and support from magnetic fields \citep{1966MNRAS.132..359S,1976ApJ...207..141M}. 

Numerical experiments investigating turbulence and magnetic fields suggest that, while magnetic support found in MHD simulations can slow the rate of star formation 
compared to hydrodynamics simulations, neither turbulence nor magnetic support is sufficient to limit the small scale star formation rate to 1-2\% per free fall time
\citep{2010ApJ...709...27W,2011MNRAS.410L...8C,2011ApJ...730...40P,2012ApJ...754...71K,2014MNRAS.439.3420M,2015ApJ...808...48B,2017ApJ...838...40M}.


\section{Review of Analytic Theories of Star Formation} \label{sec:sf_review}
I summarize analytic theories of star formation, and what each individual step has indicated for our understanding of this process.
Beginning in the late 1960's, star formation theory with the idea that clumps destined to form stars started in hydrostatic equilibrium (HSE). The gas in a cloud was assumed to be supported against its self gravity via thermal pressure support. 
The idea, expressed by Shu among others, was that something, perhaps the conversion of atomic H to molecular H$_2$, removed the thermal support at small radii, and the cloud then collapsed to form a star.

\subsection{\citet{1969MNRAS.145..271L} and \citet{1969MNRAS.144..425P}}
\citet{1969MNRAS.145..271L} and \citet{1969MNRAS.144..425P} first put forth the idea that the collapse of this core could be described by a self-similar solution. 

Larson, in his numerical experiments, began with a sphere of gas with  uniform density and temperature initially independent of radius. 
The initial velocity was zero everywhere, but clearly the gas was not in hydrostatic equilibrium. His numerical integration showed that the density approached a power law, density $\rho\sim 1/r^2$, with a core of constant density at small (and shrinking) radii. At a fixed time, at small radii, the region where $\rho \propto r^0$ is called the core. 

These initial conditions are somewhat artificial, leading \citet{1977ApJ...214..488S} to wonder how the system could have zero velocity while experiencing a non-zero net force.


%At small radii, larson finds that the density goes to a constant, that is forms a core. The velocity goes like 2/3 r (in a given moment in time) and so, we see that the infalling gas is slowing down, as it approaches the core. 
%This is seen larson finds that the density goes to a constant.

\citet{1969MNRAS.145..271L} finds that at any given instance in time, at small radii the infall velocity decreases with decreasing radius.
He also finds that the density approaches a constant at small radii, with the result that $\dot{M} \propto r^3 $. 
The mass accretion rate decreases with decreasing radius, at small radii.

It is difficult to extract $\dot M$ at large radii from this early published numerical work.
However, the numerics were enough to motivate the authors to seek a self-similar analytic solution to describe the results.

For example, \citet{1969MNRAS.145..271L} presents a simple self-similar analytic model. In contrast to his numerical model, which had a small infall velocity at large radii, in his analytic model  the velocity approaches a constant value, with a Mach number of 3.28, at large radii.
The density $\rho \propto c_s^2 / r^2$, and this provides a mass accretion, $\dot{M} \propto c_s^3 / G$. In words, the mass accretion at large radii does not depend upon the radius. 
Note that this asymptotic result apples well inside the outer boundary.

\citet{1969MNRAS.145..271L} noted the fact that at small radii, the solution was such that the pressure gradient is a reasonable fraction (0.6) of gravity (see his equation C10). 


\subsection{\citet{1977ApJ...214..488S}} \label{subsec:Shu_review}
\citet{1977ApJ...214..488S} found that the solution described by \citet{1969MNRAS.145..271L} and \citet{1969MNRAS.144..425P} was premised on a physically artificial initial condition.
One of the issues with the model was the assumption that the flow at large radii be directed inwards at velocity ${\rm v} \rightarrow 3c_s$ as $r \rightarrow \infty$. 
Shu argued that this property is non-physical, nor was it apparent that the flow could stably transition from supersonic to subsonic motion, and then to rest. 
In addition, only specific initial and boundary conditions could lead to the Larson-Penston flow.
\citet{1977ApJ...214..488S} went on to show that the flow during collapse does exhibit self-similar properties described in \citet{1969MNRAS.145..271L}.
To rectify the issue with the boundary condition, \citet{1977ApJ...214..488S} explicitly assumed that the cloud is hydrostatic for radii larger than $r = c_s t$. 
At $t=0$ a perturbation causes the central region to collapse. 
For regions where $r < c_s t$, the layers find that their pressure support has disappeared and begin to fall inwards. 
This collapse expands outwards in time, leading to the phrase ``inside-out collapse''. 
Following this solution, \citet{1977ApJ...214..488S} estimated the accretion rate onto 
stars by assuming that stars form from hydrostatic cores supported by thermal gas pressure. 
The accretion rate in his model was independent of time, given by $\dot{M} = m_0c_s^3/G$, where $c_s = (k_b T / \mu)^{1/2}$  is the sound
speed in molecular gas, and $m_0 = 0.975$. 
\citet{1977ApJ...214..488S} predicted a maximum accretion rate of $\sim 2\times10^{-6} M_\odot\,\yr^{-1}$.

\subsection{\citet{1992ApJ...396..631M}}%{Theory of Turbulent Pressure Support}
Shu's predicted maximum  accretion rate is too small to explain the origin of massive ($M_*\sim50-100M_\odot$) O stars, which have lifetimes $\lesssim 4\times10^6\yrs$, but would take 50 Myrs to grow, according to his model.
\citet{1992ApJ...396..631M}  overcame the difficulty with low accretion rates by noting that high mass star forming regions had linewidths much larger than thermal motions could produce. 
They continued to assume the initial condition was that of a hydrostatic core that is supported by turbulent pressure. 
\citet{1992ApJ...396..631M} also followed \citet{1977ApJ...214..488S} in assuming that there is an expanding collapse wave, but that the velocity was not $c_s$, but rather the sum: 
%
\be
\sigma^2 = \sigma_T^2 + \sigma_{NT}^2, 
\ee
%
where $\sigma_T$ is the thermal velocity and $\sigma_NT$ is the turbulent or nonthermal velocity. 

For radii where the non-thermal velocity dominates the line-width, 
%
\be
\dot M_* = m_o v_T^3/G,
\ee
%
where $v_T$ is the velocity inferred from the observed linewidth. 
This is larger than Shu's estimate by a factor of the Mach number cubed, ${\cal M}^3$; since the Mach number is often of order 5 or higher, the predicted accretion rates are two orders of magnitude larger than those predicted by \citet{1977ApJ...214..488S}.

They also assumed that the turbulence is static and unaffected by the collapse, i.e. $\vt (r,t) \rightarrow \vt (r)$.  
\citet{1997ApJ...476..750M} and \citet{2003ApJ...585..850M} made similar assumptions, and found the same result.  

Collectively, these models, \citep{1977ApJ...214..488S,1992ApJ...396..631M,1997ApJ...476..750M,2003ApJ...585..850M}, 
are referred to as inside-out collapse models; the collapse starts at small radii (formally at $r=0$ in the analytic models) 
and works its way outward, at the assumed propagation speed ($c_s$ or $\vt (r)$). 
At any given time, the infall velocity and mass accretion rate both decrease with increasing radius $r$. 
The analytic models assume the existence of a self-similarity variable $x = r/vt$, where $v=c_s$ in 
\citet{1977ApJ...214..488S} or the turbulent velocity $\vt (r)$ in \citet{1992ApJ...396..631M,1997ApJ...476..750M,2003ApJ...585..850M}.  
These models predict velocity and mass accretion profiles very different than those seen 
in the simulations of \citet{2015ApJ...800...49L}.


\section{Current Numerical simulations and Observations} \label{current_obs}
Numerical experiments investigating the effects of turbulent pressure and magnetic fields suggest that neither turbulence nor magnetic support is sufficient 
to reduce the rate of star formation to $\eta\approx 0.02$ on small scales
\citep{2010ApJ...709...27W,2011MNRAS.410L...8C,2011ApJ...730...40P,2012ApJ...754...71K,2014MNRAS.439.3420M}. 
Feedback from radiative effects and protostellar jets and winds may be able to explain the low star formation rate, 
but the impact of these forms of stellar feedback remains uncertain despite recent progress
\citep{2010ApJ...709...27W,2014MNRAS.439.3420M,2015MNRAS.450.4035F}.

The following subsections are intended to give the reader a general feel for the state of the field. 
It is not an extensive, much less complete review.

\subsection{Krumholz \& McKee - The Berkeley Group}
The group that I refer colloquially to as the Berkeley Group have typically simulated a turbulent core model.
In \citet{2003ApJ...585..850M}, they present evidence that massive-star forming regions are supersonically turbulent, 
and show that the molecular cores out of which individual massive stars form are as well.
The motivation behind this model is the recognition that massive stars form in regions of very high pressure and density.

The Berkeley group performed radiation hydrodynamic simulations using ORION,
\citep{2009Sci...323..754K}, and found that the radiation pressure emitted by a simulated massive prestellar core does not halt mass accretion. 

\citet{2014MNRAS.439.3420M}, actually has some of the same results that I present in this paper, namely that $M_*(t) \propto t^2$. 
Figure 7 presents the star formation efficiency vs free-fall time. While the authors fit a tangent line to determine the slope, their data presents the $M \propto t^2$ relation that I have found. 

\citet{2018MNRAS.473.4220L} create simulations using magnetic fields, radiative and outflow feedback. 
They present the SFE vs free-fall time in their Figure 9 and note that their data also shows the same $M \propto t^2$ relation that I find in Chapter \ref{ch:hydro}. 
In addition, they also find in their Figure 4, that the density goes to an $\rho \propto r^{-3/2}$.



%In 2005 Mark Krumholz and Christopher McKee derived an analytic prediction for the star formation rate in a variety of environments \citep{2005ApJ...630..250K}. 
%Their predictions were based on three premises: 
%``(1) star formation occurs in virialized molecular clouds that are supersonically turbulent; (2) the density distribution within these clouds is lognormal, as expected for supersonic isothermal turbulence; and (3) stars form in any subregion of a cloud that is so overdense that its gravitational potential energy exceeds the energy in turbulent motions.''\citep{2005ApJ...630..250K}
%\citet{2005ApJ...630..250K} then used their theory to derive the Kennicutt-Schmidt law from first principles. 
%\citet{2005ApJ...630..250K}

\subsection{Klessen \& Mac Low Group} 
%\citet{2010A&A...512A..81F} https://www.aanda.org/articles/aa/pdf/2010/04/aa12437-09.pdf
One of the major questions in implementing turbulence in numerical simulations is how one decides to drive it. 
There are two ways in which one can drive turbulence, either solenoidally, think stirring cream into a cup of coffee, 
or via compression, think clapping your hands together. 
Depending upon the choice of driver, or even mixture of the two, one might drastically change the simulated results. 
So, what choice of driving method best models the turbulence seen in the ISM, and in GMCs in particular?
This question was resolved when \citet{2010A&A...512A..81F} showed that irrespective of the type of forcing, 
they found velocity dispersion-size relations consistent with observations, and independent numerical simulations. 
They found that the forcing they used did not change the physical relations at smaller scales. 

\citet{2010A&A...512A..81F} achieved this by using high resolution simulations of pure solenoidal and pure compressive turbulence. 
In addition, they ran several lower resolution simulations with varying mixtures of the drivers.
\citet{2010A&A...512A..81F} found that, `` although likely driven with mostly compressive modes on large scales, turbulence can behave like solenoidal turbulence on smaller scales.''
The turbulent cascade had reached equipartition, of 2:1 solenoidal to compressive. 

Previous work by the group had found that under typical GMC conditions global collapse could be prevented, 
but strong shocks would become gravitationally unstable and collapse to presumably form stars \citep{2000ApJ...535..887K}.

%\citet{2004ApJ...605..800L} have results that are consistent with the predictions of Padoan and Nordlund, whom I'll discuss next.

%\subsection{Nordlund \& Padoan}
%
%\citet{2011ApJ...730...40P} presented a new model of star formation that depends on the relative importance of gravitational, turbulent, magnetic, and thermal energies.
%They presented self-gravitating MHD simulations, and show that turning off magnetic fields leads to an increase in the SFR by a factor of three \citep{2011ApJ...730...40P}. 
%

\subsection{\citet{2015ApJ...800...49L}} \label{Lee15_model}

\citet{2015ApJ...800...49L} showed that, in simulations with no feedback, the star formation efficiency on
parsec scales is not constant in time.
This is in contrast to previous work, which had implicitly assumed that the star formation rate on small scales was constant.
In particular, many authors have assumed that the star formation rate in their simulations of a GMC (or smaller cloud or part of a cloud) was given by equation (\ref{eq:intro_KS_law}), where $\epsilon$ was assumed to be constant.
\citet{2015ApJ...800...49L} showed that $\epsilon \propto t$, which implies that $M_* \propto t^2$, where $M_*$ is the total stellar mass.

\citet{2015ApJ...800...49L} emphasized that the star formation efficiency on 
parsec scales is nonlinear 
in time, i.e., $\epsilon \propto t \rightarrow M_* \propto t^2$, on small scales, where $M_*$ is the total stellar mass. 
%Magnetic fields slowed the initial star formation rate somewhat, but did not change the $M_*(t)\sim t^2$ scaling. 
Using a detailed numerical simulation, they showed that this nonlinear star formation 
rate is driven by the properties of collapsing regions.
In particular, they demonstrated that the turbulent velocity near or in collapsing regions follows 
different scaling relations than does turbulence in the global environment, which follows
Larson's law, $\vt (r)\sim r^{1/2}$ \citep{1981MNRAS.194..809L}. 
They also showed that the density PDF 
is not log-normal, but rather develops a power law to high density.
The power law tail to the  PDF  was hinted at in much earlier simulations by \citet{2000ApJ...535..869K} and shown convincingly, 
as well as explained, by \citet{2011ApJ...727L..20K}. 

The increasing rate of star formation found by \citet{2015ApJ...800...49L} is important 
in that it may provide an explanation for the observed range in star formation rates on 
small scales.  It suggests that the star formation rates on small scales 
vary in part because of the age of the star forming region; slow star forming 
regions, with very low instantaneous efficiencies, will ramp up their stellar production over time. If this result
can be firmly established, it will highlight the need for a form of very rapid 
feedback. In particular, since the dynamical time in massive star forming regions 
is much smaller than the time delay of $\sim 4{\,\rm Myrs}$ between the start of star 
formation and the first supernovae, rapid star formation on small scales would 
have to be halted by some form of feedback other than supernovae. 

The simulations of  \citet{2015ApJ...800...49L} explicate the link between the rate of star formation with the gravitational collapse of high density regions, which  is  an analytically well studied problem. 

\subsection{\citet{2015ApJ...804...44M}} \label{MC15_model}
%Motivated by this, \citet[][hereafter MC15]{2015ApJ...804...44M} developed a new 1-D model of spherical collapse that treats the turbulent velocity, $v_T$, as a dynamical variable.
%%In essence, they extended the classic analysis of \citet{1977ApJ...214..488S} and \citet{1992ApJ...396..631M}; see also \citet{1997ApJ...476..750M,2003ApJ...585..850M}.
%%
%Following earlier work \citep{1977ApJ...214..488S,1992ApJ...396..631M,1997ApJ...476..750M}, MC15 reduced the fluid equations by assuming spherical symmetry, but did not assume a fixed equation of state.

Motivated by this discrepancy between observation and the current analytical models,
\citet{2015ApJ...804...44M}, hereafter MC15, developed a 1-D model of
spherical collapse that treats the turbulent velocity, $\vt$, as a dynamical variable and
does not assume that the initial condition is a hydrostatically supported region. 
They used the results of \citet{2012ApJ...750L..31R} on compressible turbulence; the evolution 
of the turbulent velocity in a collapsing (or expanding) region is described well by the following equation:
%
\be
\frac{\partial \vt}{\partial t} + \ur \frac{\partial \vt}{\partial r} 
+ \left( 1 + \eta \frac{\vt}{\ur} \right) \frac{\vt \ur}{r} = 0
\label{eq:hydro_Robertson}
\ee
%
The first two terms are the Lagrangian derivative, and $\ur$ is the radial infall velocity. The first term in the brackets 
describes the turbulent driving produced by the infall, while the second 
is the standard expression for the turbulent decay rate;  $\eta$ is a dimensionless constant of order unity.

MC15 used this in place of an energy equation. Together with the
equations for mass continuity and momentum, equation (\ref{eq:hydro_Robertson}) gives a
closed set of equations that can be solved in spherical symmetry
numerically. In addition, they were able to analytically show that the
results of their calculations gave density and velocity profiles that
appear to be in line with both recent numerical calculations
\citep{2015ApJ...800...49L} and observations (e.g.,
\citealt{1995ApJ...446..665C,1997ApJ...476..730P}).

To summarize, MC15's major results were:
\begin{itemize}
\item The gravity of the newly formed star introduces a physical scale into the problem, which MC15 called the stellar sphere of influence, $r_*$. This is an idea familiar from galactic dynamics. 
The radius $r_*$ is where the local dynamics transitions from being 
dominated by the mass of the gas to being dominated by the mass of the star. 
As a result, the character of the solution, in particular that of the velocity, differs dramatically between $r<r_*$ and $r>r_*$. 
The existence of this physical scale modifies the form of the self-similarity on which inside-out theories rely.

\item The small scale density profile is an attractor solution.  MC15 showed numerically and argued analytically that at small scales, the density profile is an attractor solution. 
In particular, MC15 showed the density profile asymptotes to: 
%
\be
\rho(r,t)=
\begin{dcases}
\rho(r_0)\left({\frac{r}{r_0}}\right)^{-3/2}, & r<r_*(t)\\
\rho(r_0,t)\left({\frac{r}{r_0}}\right)^{-k_\rho}, \ k_\rho\approx1.6-1.8 & r>r_*(t).
\end{dcases}
\label{eq:hydro_density}
\ee
%
where $r_0$ is some fiducial radius. 
\item The existence of $r_*$ implies that the infall and turbulent velocities have different scaling for $r<r_*$ and $r>r_*$.  In particular, MC15 showed
%
\be
u_r(r,t), v_T(r,t) \propto
\begin{dcases}
\sqrt{\frac{GM_*(t)}{r}} \sim r^{-1/2} & r<r_*(t)\\
\sqrt{\frac{GM(r,t)}{r}} \sim r^{0.2} & r>r_*(t),
\end{dcases}
\label{eq:hydro_infall_behavior}
\ee
%
Thus the scaling of the turbulent velocity differs from that predicted by Larson's law ($\propto r^{1/2}$) inside the sphere of influence. In other words, the turbulent velocity in  massive star forming regions will deviate from Larson's law, which has long been observed, but without theoretical explanation. 
\item The stellar mass increases quadratically with time.  This result  arises naturally from the attractor solution nature of the density profile at small $r$, Equation (\ref{eq:hydro_density}), and the scaling with Keplerian velocity for the turbulent and infall velocities at small $r$, Equation (\ref{eq:hydro_infall_behavior}).

The mass accretion rate: 
%
\be
\dot M(r,t)=
\begin{dcases}
4\pi R^2\rho(R)u_r(r,t), \sim t\,r^{0} & r<r_*\\
4\pi R^2\rho(R)u_r(r,t) \sim t^0\,r^{0.2} & r>r_*.
\end{dcases}
\label{eq:hydro_Mdot_behavior}
\ee
%
\end{itemize}

MC15's predictions for $r<r_*$ could not be checked using the simulations
of \citet{2015ApJ...800...49L} as those fixed grid simulations were
too coarse.  


\subsection {Galaxy Scale Simulations}

Until very recently, galaxy-scale or larger (cosmological) simulations 
were not able to reproduce the Kennicutt-Schmidt relation. 
Nor did the cosmological runs reproduce correctly the mass of stars in galaxies of a given 
halo mass, despite including supernova and other forms of feedback, 
e.g., \citet{2010MNRAS.404.1111G,2010Natur.463..203G,2011MNRAS.410.2625P}.
To overcome this low resolution driven problem, \citet{2011MNRAS.417..950H,2012MNRAS.421.3522H}
performed high resolution (few parsec spatial, few hundred solar mass particle masses) simulations 
of isolated galaxies, modeling both radiative and supernovae feedback (among other forms). 
They recovered the Kennicutt-Schmidt relation, a result that they 
showed was independent of the small-scale star formation
law that they employed. 
The simulations in the second paper also generated galaxy 
scale outflows or winds, removing gas from the disk, thus making it unavailable for star 
formation. 
When the feedback was turned off, the star formation rate soared, demonstrating that
in the simulations at least, feedback was crucial to explaining the Kennicutt-Schmidt relation, and
the outflows. 
Simulations including supernovae but lacking the radiative component of the feedback
did not exhibit strong winds and so overproduced stars.

Cosmological simulations employing unresolved (or ``sub-grid") models for both 
radiative and supernovae feedback are now able to reproduce the halo-mass/stellar mass relation (e.g.,
\citealt{2013MNRAS.434.3142A,2014MNRAS.445..581H,2015ApJ...804...18A}). Again, 
these simulations {\em require} 
stellar feedback to drive the winds that remove gas from the disk, so as to leave the observed mass of stars behind. 


\section{What this Dissertation does not cover}
There are a number of groups whose observational, statistical, and analytical, contributions and theories I have not covered.
I do not intend to attempt to create an extensive review.
%Magnetic fields are about in equipartition \fraction{$B^2$}{$8 \pi$} \approx \vt^2$


\subsection{Missing physics} \label{subsec:missing_physics}
Our current understanding of star formation suggests that the effects of magnetic 
fields, radiation from stars, and the heating and cooling rates of the gas can all have significant effects on both the rate of star formation and the initial 
mass function (IMF) of the stars. 
We do not include any of this physics in the simulations described in this paper. 
The simulations presented in this dissertation are focused upon turbulence in hydrodynamic simulations and the effects of protostellar jets. 

It is often argued that the turnover in the IMF, somewhere between $0.2$ and $0.6\,M_\odot$, 
is associated with the thermal Jeans mass of the gas in the collapsing region. 
There have been other proposed explanations, for instance, 
\citet{2002ApJ...576..870P} found that the IMF depends upon turbulent fragmentation, however, this is hard to understand because turbulence does not have a scale associated with it. 
\citet{2016MNRAS.460.3272K} state that on small scales the dominant mechanism limiting fragmentation of gas is the thermal pressure. 
They state that this thermal pressure is influenced by stellar radiation: as the gas fragements and collapses the opacity of the gas grows trapping the radiation from the young stars, leading to a change in temperature of the gas, increasing the thermal pressure. 
This picks out a mass scale.
If so, then our use of an isothermal equation of state suggests that the IMF found in our simulations is likely to be in error. 
However, as Figures \ref{fig:hydro_Sphere_of_influence_quad2_well_before}-\ref{fig:hydro_Sphere_influence_quad2_end_time} and \ref{fig:hydro_velocity_avg} show, 
both $\ur$ and $\vt$ exceed $c_s$, except at the earliest times ($\sim 100,000\yrs$ before a star forms), and then only for $r\lesssim0.1\pc$, 
so that the gas pressure does not dominate the dynamics in most regions and most of the time. 
%%TODO - what am I trying to say here.
%Of course we do include the effects of gas pressure, so even in those regions and those times, 
%our simulations capture the dynamical effects aside from, as we have just said, fragmentation effects on the smallest scales. 

We have undertaken and made some preliminary analyses of magnetohydrodynamic simulations, which we will report on in future publications; 
as seen by other authors, we find that magnetic fields slow the star formation rate. 
But the runs of density and velocity have the same qualitative form in our MHD simulations as in the hydro runs presented here, and the MHD runs also give $M_*(t)\sim t^2$. 

%TODO - Phil asks why is this line here.
%Feedback from protostellar outflows are seen to slow the rate of star formation, e.g., \citet{2010ApJ...709...27W,2015MNRAS.450.4035F}. 
%But those authors, like us, find that $dM_*/dt$ increases with time even in runs that include outflows. 

Radiative feedback will also affect both the IMF and, for massive enough stars, the dynamics of the collapse at late times (after massive stars have formed). 
Massive stars can emit high velocity winds, up to $3000 \kms$, as these winds leave the star they create shocks when the wind runs into the surrounding gas.
Due to the collision of the wind and surrounding gas, the gas heats up to $10^8 - 10^9 \K$. 
The pressure associated with these high temperatures will push the surrounding gas away from the star.

All the figures we show present results for stars with masses no larger than about $4 M_\odot$. To estimate the effects of radiation, we compare the force from the Reynold stress $F_T=4\pi r^2 \rho v_T^2$, to the radiation force $L/c$. 
From Figure \ref{fig:hydro_velocity_avg}, the (averaged over many stars) $v_T$ is slightly in excess of $1\kms$ at $r=0.01\pc$, while from any of the density figures the density is $\rho\approx5\times10^{-18} {\rm g/cm}^3$. The force from Reynold stress is then $F\approx4\times 10^{26}$ dynes. The luminosity of a 4 solar mass star on the zero age main sequence is $L\approx 2\times10^{36}\ergs$ \citep{1992A&AS...96..269S}, so the radiation force $L/c \approx 3\times10^{25}\,{\rm dynes}$, about a 10\% effect.  The force from Reynolds stress increases outward, see Figure \ref{fig:hydro_dPdr_avg}, so this statement holds at larger radii as well. 

Thus we expect that the effects of radiation pressure are not particularly significant in the situations we report;  the run of density and infall velocity, and hence the $M_*(t)\sim t^2$ scaling should not be affected, at least up to the times we are reporting on.  We note, however, that this estimate neglects the effect of radiative or ionization heating which is an important feedback mechanism.

Simulations including radiative feedback support this simple analysis. Figure 15 of  \citet{2014MNRAS.439.3420M} shows that in their simulations, which include feedback from both protostellar outflows and radiation (as well as magnetic fields), the stellar mass increases as the square of the time, up to masses of ~4.5 solar masses. Earlier work by the Berkeley group found similar results, forming stars with 10 solar masses, with $M_*(t)\sim t^2$ even for such massive stars, see Figure 13 of \citet{2012ApJ...754...71K}. Their simulations included radiative effects, but no protostellar winds. 


This dissertation describes my contributions to various parts of our current understanding of the dynamics of star formation. 
This chapter gives a brief overview of the theory behind star formation and notes gaps in our current understanding. 
Ch. \ref{ch:hydro} discusses the initial set up for the majority of my simulations, and the dynamics of gravitational collapse with no feedback or other delaying aspects.
Ch. \ref{ch:jet} discusses the effects that protostellar and jet feedback have on the stellar mass accretion rate for a cluster of stars.
Finally, in Ch. ~\ref{ch:conclusions}, I summarize the results obtained in this dissertation.

\end{document}

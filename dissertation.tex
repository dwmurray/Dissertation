\documentclass[12pt,notitlepage]{report}

% Included in Maddy's example to me.
\usepackage{graphicx,vmargin,fancyhdr,amsfonts,amsmath,amsthm,amssymb,mathrsfs}
\usepackage{uwmthesis2}
\usepackage{natbib}
\usepackage{subfigure}
\bibpunct{(}{)}{;}{a}{}{,}
\usepackage{color}
\usepackage{subfiles}
\usepackage[titletoc,toc,title]{appendix}
\usepackage{epigraph}
\usepackage{longtable}
\usepackage{aas_macros}

% From my MNRAS papers.
%\usepackage{newtxtext,newtxmath}
%\usepackage{ulem}
%\usepackage[T1]{fontenc}
%\usepackage{ae,aecompl}

\usepackage{mathtools}

%\sloppy %From Maddy's example. %Not great to use, so commented out by me. May require this.

%------------Set Margin{left}{top}{right}{bottom}{foot notes}{headers etc}{}----
\setmarginsrb{1.0in}{1in}{1in}{1.0in}{0pt}{0.in}{0pt}{1cm}

%------------Set page numbering style---------------
\pagestyle{fancy} \lhead{} \chead{} \rhead{} \lfoot{} \cfoot{\thepage} \rfoot{}

%------------Previous users macros I left in------------
\renewcommand{\headrulewidth}{0pt}
\renewcommand{\footrulewidth}{0pt}
\fancyhfoffset[]{0.25In}

%%%%% number equations by section %%%%%%%%
\makeatletter
\@addtoreset{equation}{section}
\makeatother
\renewcommand{\theequation}{\thesection.\arabic{equation}}
%%%%%%%%%%%%%%%%%%%%%%%%%%%%%%%%%%%%%%%%%%%

% macros for thesis % I converted to newcommand, rather than define
\newcommand{\tit}{TURBULENT COLLAPSE OF GRAVITATIONALLY BOUND CLOUDS}
\newcommand{\aut}{\large Daniel W. Murray}
\newcommand{\mjrprof}{Philip Chang}
\newcommand{\inh}{\today}
%\def\tit{TURBULENT COLLAPSE OF GRAVITATIONALLY BOUND CLOUDS}
%\def\aut{{\large Daniel W. Murray}}
%\def\inh{\today}

%Macros for Setting Spacing Between Lines
\newcommand{\sglspc}{\setstretch{1.1}}
\newcommand{\dblspc}{\setstretch{1.6}}

% math macros
\newcommand{\be}{\begin{equation}}
\newcommand{\ee}{\end{equation}}
\newcommand{\bfig}{\begin{figure}}
\newcommand{\efig}{\end{figure}}

\newcommand{\lp}{\left(}
\newcommand{\rp}{\right)}

% Common Velocities I use
\newcommand{\ur}{\mathbf{u_r}}
\newcommand{\vt}{\mathbf{v_T}}
\newcommand{\vkep}{\mathbf{v_{\rm K}}}
\newcommand{\vphi}{\mathbf{v_\phi}}
\newcommand{\vff}{\mathbf{v_{\rm ff}}}
\newcommand{\vrms}{\mathbf{v_{\rm rms}}}
\newcommand{\cs}{c_{\rm s}}

\newcommand{\etaeff}{\eta_{\rm ff}}
\newcommand{\tff}{\tau_{\rm ff}}
\newcommand{\tdyn}{\tau_{\rm dyn}}

%--------------------- Units ----------------------------%
% Solar
\newcommand{\Msun}{{\,\rm M}_\odot}
\newcommand{\Rsun}{{\,\rm R}_\odot}
\newcommand{\Lsun}{{\,\rm L}_\odot}

% Length
\newcommand{\kpc}{\,\rm kpc}
\newcommand{\pc}{\,\rm pc}
\newcommand{\AU}{\,\rm AU}
\newcommand{\cm}{\,\rm cm}

% Mass
\newcommand{\g}{\,\rm g}

% Time
\newcommand{\Myrs}{\,\rm Myrs}
\newcommand{\yrs}{\,\rm yrs}
\newcommand{\yr}{\,\rm yr}
\newcommand{\s}{\,\rm s}

% Velocity
\newcommand{\kms}{\,\rm km\,s^{-1}}
\newcommand{\cms}{\,\rm cm\,s^{-1}}

% Energy
\newcommand{\ergs}{\,\rm erg\, s^{-1}}

% HII regions
\newcommand{\HII}{{\rm H}_{\rm II}}

%------------------------ Figure Commands -------------------------------%
\newcommand\plotonesmall[1]{\begin{center} \includegraphics[width=0.5\textwidth]{#1} \end{center}}
\newcommand\plotonebig[1]{\begin{center} \includegraphics[width=0.8\textwidth]{#1} \end{center}}
\newcommand\plotone[1]{\begin{center} \includegraphics[width=0.6\textwidth]{#1} \end{center}}
\newcommand\plottwo[2]{\begin{center} \includegraphics[width=0.5\textwidth]{#1}\includegraphics[width=0.5\textwidth]{#2} \end{center}}
\newcommand\plotthree[3]{\begin{center} \includegraphics[width=0.3\textwidth]{#1}\includegraphics[width=0.3\textwidth]{#2}\includegraphics[width=0.3\textwidth]{#3} \end{center}}

%Hydro Specific, should clean and remove from ms.
\newcommand{\partA}{particle A }
\newcommand{\PartA}{Particle A }
\newcommand{\partB}{particle B }
\newcommand{\PartB}{Particle B }
%\newcommand{\omhat}{\hat{\Omega}}
%\newcommand{\phat}{\hat{p}}
%\newcommand{\hplus}{h_+}
%\newcommand{\hcross}{h_{\times}}
%\newcommand{\infint}{\int_{-\infty}^{\infty}}
%\newcommand{\bb}{\begin{bmatrix}}
%\newcommand{\eb}{\end{bmatrix}}
% \DeclareMathOperator{\Tr}{Tr}
%\newcommand{\R}[1]{\textcolor{red}{#1}}

\begin{document}

\title{\tit}
\author{\aut}
\majorprof{\mjrprof}%{Philip Chang}
\submitdate{May 2018}
\degree{Doctor of Philosophy}
\program{Physics}
\copyrightyear{2018}
\majordept{Physics}

\havededicationfalse
%\dedication
\haveminorfalse
\copyrightfalse

\doctoratetrue
%\figurespagetrue%Default is on.
\tablespagefalse

% A the manuscript title page
\manuscriptp

\pagenumbering{roman}
\setcounter{page}{1}
\pagestyle{plain}

% B the Official Approval page
%\officialaprovp


\setcounter{page}{2}

%---------------------------------Abstract-----------------------------------------
\begin{center} 
\sglspc

{\Large \bf ABSTRACT}\\
\tit\\
%TURBULENT COLLAPSE OF GRAVITATIONALLY BOUND CLOUDS\\

\

The University of Wisconsin--Milwaukee, May \number\year\\
Under the Supervision of Professor \mjrprof
\end{center}
\ \\
\dblspc %set double spacing
%--------Abstract Text-------------------------
\vskip -1cm

%This dissertation presents the story of star formation. 
In this dissertation, I explore the time-variable rate of star formation, using both numerical and analytic techniques. 
I discuss the dynamics of collapsing regions, the effect of protostellar jets, and development of software for use in the hydrodynamic code RAMSES.
I perform high-resolution adaptive mesh refinement simulations of star formation in self-gravitating turbulently driven gas. 
I have run simulations including hydrodynamics (HD), and HD with protostellar jet feedback.
Accretion starts when the turbulent fluctuations on largescales, near the driving scale produce a converging flow.
I find that the character of the collapse changes at two radii, the disk radius $r_d$, and the radius $r_*$ where the enclosed gas mass exceeds the stellar mass. 
This is the first numerical work to show that the density evolves to a fixed attractor, $\rho(r,t ) \rightarrow \rho(r)$, for $r_d<r<r_*$; mass flows through this structure onto a sporadically gravitationally unstable disk, and from thence onto the star. 
The total stellar mass $M_*(t)\sim (t-t_*)^2$, where $(t-t_*)^2$ is the time elapsed since the formation of the first star. 
This is in agreement with previous numerical and analytic work that suggests a linear rate of star formation.
I show that protostellar jets change the normalization of the stellar mass accretion rate, 
but do not strongly affect the dynamics of star formation in hydrodynamics runs. 
In particular, $M_*(t) \propto (1 - f_{\rm jet})^2 (t-t_*)^2$ is the fraction of mass accreted onto the protostar, 
where $f_{\rm jet}$ is the fraction ejected by the jet. 
For typical values of $f_{\rm jet} \sim 0.1 - 0.3$ the accretion rate onto the star can be reduced by a factor of two or three. 
However, I find that jets have only a small effect (of order 25\%) on the accretion rate onto the protostellar disk (the ``raw'' accretion rate). 
In other words, jets do not affect the dynamics of the infall, but rather simply eject mass before it reaches the star.
Finally, I show that the small scale structure --- the radial density, velocity, and mass accretion profiles --- are very similar in the jet and no-jet cases.

\endabstract 
\newpage

% D the OPTIONAL copy right page
\ifcopyright\copyrightpage\fi
% E the OPTIONAL dedication page
\ifhavededication\dedicationpage\fi
% F the Table of Contents
\newpage
\renewcommand\contentsname{\begin{center}  \vspace{-2.5cm}{\Large TABLE \ OF \ CONTENTS}\vspace{-1.5cm} \end{center}}
\renewcommand\listfigurename{\vspace{-2.5cm}{\begin{center}{\Large LIST \ OF \ FIGURES} \end{center}} \vspace{-0.5cm}}
\renewcommand\listtablename{\begin{center} \vspace{-2.5cm}{\Large LIST \ OF \ TABLES}\vspace{-0.5cm} \end{center}}


\tableofcontents

\afterpreface

\newpage

%---------------------------------Acknowledgments-----------------------------------------
\begin{center}
{\Large \bf ACKNOWLEDGMENTS}
\end{center}


\singlespace
\epigraph{``\emph{Sometimes the gift of an inquisitive nature to the young can be greater than that of the wisdom which comes of age.}"}{--- \textup{Abbot Mordalfus}, \textit{Mattimeo by Brian Jacques}}
%“Don't be ashamed to weep; 'tis right to grieve. Tears are only water, and flowers, trees, and fruit cannot grow without water. But there must be sunlight also. A wounded heart will heal in time, and when it does, the memory and love of our lost ones is sealed inside to comfort us.” 
%― Brian Jacques, Taggerung
%Mice are my heroes because, like children, mice are little and have to learn to be courageous and use their wits. -- Brian Jacques
%“A little (one) can sometimes see things in others that us older ones cannot because our judgement gets clouded. —Abbot Saxtus” ― Brian Jacques, The Bellmaker

\dblspace

The inquisitiveness of many good friends and the wisdom of family and colleagues have planted me where I am.
I would not be where I am today without the constant love and support of my family. 
I would like to thank my nephew Aiden, and Phil's sons Liam and Gregory, for teaching me to approach everything with an open mind, and to question constantly.
I have a new appreciation of sprinklers and ``Mighty Machines'', due to their curiosity.

I thank my father for our many stimulating conversations about the physical world, whether it was calculations of our mpg over Tioga Pass or talk of the constellations in the heavens. % it was these conversations that set me down this road.
A special thank you to my sisters for fun times, growing up. 
My mother encouraged me to use my wits and approach the world with courage.
%I thank my sisters for the lessons learned and for guidng my branches to what I need when I need it, whether I know or not.
%My growth and the fruit it has borne is thanks to my mother. 
%My mother has tended my roots and the branches I had spread as I learned to be courageous and use my wits. 

I would like to thank Matthew Brinson, Adil Amin, and Joe Simon for the countless walks, the coffee breaks, and the occasional late night at Vintage. 

I would like to thank all the members of the Leonard E Parker Center for Gravitation, Cosmology \& Astrophysics (CGCA) and the UWM Physics Department 
for being wonderful mentors over the years. 
I want to especially thank David Kaplan and Dawn Erb for the many rousing CGCA teas, whether science was involved or one of the many festivals around the city. 

Finally, I want to acknowledge the guidance, support and encouragement from my advisor Philip Chang. 
Phil was extremely patient and persistent in shepherding me along towards independent research. 
He was instrumental in my development both professionally and personally. 

I am grateful for the opportunity to complete my dissertation as a member of the CGCA.

\newpage

%---------------------------------Preface-----------------------------------------
\begin{center}
{\Large \bf CONVENTIONS}
\end{center}
\ \\

\begin{itemize} 

\item In astronomy Hydrogen that has been ionized is refered to as $\HII$, where ``H'' is hydrogen and ``II'' is the roman numeral for 2. 

\item I use $\ur$ to indicate radial (infall) velocity.

\item I use $\vt$ to indicate the random motions velocity.

\item I use $\vphi$ to indicate the rotational velocity.

\item I use $\vkep$ to indicate the $\sqrt{GM/r}$ velocity.

\item Bold font indicates a vector or a matrix, such as $\mathbf{J}$, $\mathbf{S}$ and $\mathbf{x}$.

\end{itemize}

\dblspc %set double spacing 

\newpage

\pagenumbering{arabic}
%
\pagestyle{uwmheadings}

%ME here again, new format wants the number at cfoot, not rhead, so manually setting it here
% Unsure if uwmheadings changes anything else, so I don't want to turn it off.
%------------Set page numbering style---------------
\pagestyle{fancy} \lhead{} \chead{} \rhead{} \lfoot{} \cfoot{\thepage} \rfoot{}


%---------------------------------Ch 1: Introduction-----------------------------------------%
\subfile{./Introduction/introduction}
%\label{ch:intro}

%%---------------------------------Ch 2: Hydro/Gravitational Force Alone-----------------------------------------%
\subfile{./Hydro/hydro}
%\label{ch:hydro}

%%---------------------------------Ch 3: Protostellar/Jet Feedback------------------------------------------%
\subfile{./Jet/jet}
%\label{ch:jet}

%%---------------------------------Ch 4: Magnetic Fields-----------------------------------------%
%\subfile{./MHD/mhd}
%\label{ch:mhd}

%%---------------------------------Ch 5: Conclusions-----------------------------------------%
\subfile{./Conclusions/conclusions}

\bibliographystyle{apj}
\bibliography{diss_bibliography}

%----------------------- CV --------------------------------%

\newpage
\sglspc
\begin{center}
{\bf \Large{CURRICULUM VITAE}}\\
\end{center}

\

Daniel Murray

Place of birth: Santa Monica, CA

\

\small{
\medskip\noindent
{\bfseries  EDUCATION}
\medskip

\begin{tabular}{ll}
{\it 8/2012--5/2018} & {\bf Doctor of Philosophy in Physics} \\
 			        & University of Wisconsin -- Milwaukee, Milwaukee WI \\
			        & Advisor: Dr. Philip Chang \\
			        & Awards: Research Excellence Award, Chancellor's Graduate Student Award, \\
			        & Physics Graduate Student Award \\
			        & GPA: 3.07/4.00 \\
{\it 9/2008--6/2012} & {\bf Bachelor of Science in Physics} \\
 			        & University of California Santa Barbara, Santa Barbara, CA \\
			        & GPA: 3.00/4.00 \\
\end{tabular}
}

\bigskip
\noindent{\bf RESEARCH EXPERIENCE}
\medskip

\begin{tabular}{ll}
{\it 8/2012--present} & {\bf Graduate Research Assistant} \\
				& University of Wisconsin - Milwaukee, Milwaukee WI \\
				& Advisor: Dr. Philip Chang \\
\end{tabular}	        


Employment  June 2011 - August 2011 
History:Smithsonian Astrophysical Observatory  
 Pan-STARRS collaboration under Matthew Holman
Wrote part of the data reduction pipeline, including the databasing

June 2010 – January 2011
 Las Cumbres Observatory
Research Intern under Andy Howell
Wrote and refined a program to automate the reduction of SNe data
Work resulted in two papers, published in Nature and The Astrophysical Journal.

December 2009- March 2010 
 Dr. Phil Lubin’s lab at UCSB
Research Assistant
Designed inflatable mirrors for near-infrared applications

\newpage
			        				
\bigskip
\noindent{\bf TEACHING EXPERIENCE}
\medskip

\begin{tabular}{ll}
{\it 8/2012--5/2018} & {\bf Teaching Assistant and Lecturer}\\		
				& University of Wisconsin -- Milwaukee, Milwaukee, WI \\
				& {\it Discussion section leader for introductory and advanced physics courses} \\
				& Led discussion sections for both introductory and \\
                                & advanced undergraduate courses \\
			        				
\end{tabular}

%
\bigskip
\noindent{\bf FELLOWSHIPS \& AWARDS}
\medskip

%
\begin{tabular}{ll}
%{\it 2013, 2014} & Distinguished Dissertation Fellowship Nominee (UWM)\\
{\it 2018} & Department of Physics Papastamatiou Scholarship (UWM)\\
%
\end{tabular}

\bigskip
\noindent{\bf INVITED TALKS}
\medskip

%\begin{tabular}{ll}
%{\it 5/2017} & {\bf Midwest Magnetic Fields Workshop} \\
%		& University of Wisconsin-Madison, Madison, WI \\
%		& Presented on dissertation research.
%		
%\end{tabular}

\bigskip
\noindent{\bf CONFERENCES AND PRESENTATIONS}


\begin{itemize}

%\item[]{\bf Iowa High Performance Computing Summer School} Iowa City, IA July 2013. Learned the fundamentals of HPC, includeing OpenMPI and CUDA

\item[]{\bf Princeton’s Institute for Advanced Study Prospects in Theoretical Physics 2016}, Princeton, NJ July 2016. Oral presentation on the effects of turbulence on gravitational collapse.

\item[]{\bf Graduate Student Research Symposium for Math, Engineering, and the Natural Sciences}, Milwaukee, WI, October. 2016. Poster presentation on the effects of turbulence on gravitational collapse.

\item[]{\bf Midwest Magnetic Fields 2017 Workshop}, Madison, WI, May. 2017. Oral presentation on the effects of magnetic fields on the star formation rate for massive star forming regions.

\end{itemize}

\bigskip
\noindent{\bf PUBLICATIONS}

\textbf{Primary Publications}

\begin{itemize}

%\item[] \textbf{M. Wade}, J. Creighton, E. Ochsner, and A. Nielsen. ``Advanced LIGOÕs ability to detect apparent violations of the cosmic censorship conjecture and the no-hair theorem through compact binary coalescence detections." Phys. Rev. D 88 (2013) 083002.

%\item[] \textbf{M. Wade}, X. Siemens, et al. ``Making $h(t)$ in Advanced LIGO". (in prep.)
\item[] \textbf{D. Murray}, and P. Chang. ``The Effects of Magnetic Fields on Turbulent Collapse''. (in prep.)

\item[] \textbf{D. Murray}, S. Goyal, and P. Chang. ``The Effects of Protostellar Jet Feedback on Turbulent Collapse.'' MNRAS 2018 arXiv: 1710.09415

\item[] M. Holman, M. Payne, W. Fraser, P. Lacerda, M. Bannister, M. Lackner, Y. Chen, H. Lin, K. Smith, R. Kokotanekova, K. Chambers, S. Chastel, L. Denneau, A. Fitzsimmons, H. Flewelling, T. Grav, M. Huber, N. Induni, A. Krolewski, R. Jedicke, E. Lilly, E. Magnier, Z. Mark, M. Micheli, \textbf{D. Murray}, et al. ``A Dwarf Planet Class Object in the 21:5 Resonance with Neptune.'' ApJ Letters 2017 %arXiv: 1709.05427

\item[] \textbf{D. Murray}, P. Chang, N. Murray, and J. Pittman. ``Collapse in Self-gravitating Turbulent Fluids.'' MNRAS 2017; 465:1316

\item[] E. O. Ofek, M. Sullivan, S. B. Cenko, M. M.  Kasliwal, A. Gal-Yam, S. R. Kulkarni, I. Arcavi, L. Bildsten, J. S. Bloom, A. Horesh, D. A. Howell, A. V. Filippenko, R. Laher, \textbf{D. Murray}, et al. ``An outburst from a massive star 40 days before a supernova explosion.'' Nature 2013; 494:65

\item[] A. Corsi, E. O. Ofek, D. A. Frail, D. Poznanski, I. Arcavi, A. Gal-Yam, S. R. Kulkarni, K. Hurley, P. A. Mazzali, D. A. Howell, M. M. Kasliwal, Y. Green, \textbf{D. Murray}, D. Xu, S. Ben-ami, J. S. Bloom, B. Cenko, N. M. Law, P. Nugent, R. M. Quimby, V. Pal'shin, J. Cummings, V. Connaughton, K. Yamaoka, A. Rau, W. Boynton, I. Mitrofanov, J. Goldsten ``PTF 10bzf (SN 2010ah): a broad-line Ic supernova discovered by the Palomar Transient Factory.'' ApJ 2011; 741:76

\end{itemize}

\textbf{Telegrams}

\begin{itemize}

\item[] Supernova 2010mc 2012CBET.3313....2H

\item[] Supernova 2010ko in NGC 1954. 2010CBET.2575....2P

\end{itemize}


\end{document}

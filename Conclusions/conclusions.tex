\documentclass[../dissertation.tex]{subfiles}
%\setlength{\epigraphwidth}{.7\textwidth}

\begin{document}

\chapter{Conclusions}
\label{ch:conclusions}

\singlespace
%\epigraph{``\emph{There is no real ending. It's just the place where you stop the story.}"}{--- \textup{The King}, \textit{Dune by Frank Herbert}}
\epigraph{``\emph{There is no real ending. It's just the place where you stop the story.}"}{--- \textup{The King}, \textit{Dune by Frank Herbert}}

\dblspace

I find that the star formation rate at small scales varies with time.
This has several implications: It means observers can not judge the age of a system from its run of density,
as the density has been shown reach an attractor solution.
It indicates that there must exist a form of rapid feedback, in order to disrupt star formation, as the 4 million year wait for supernovae is too long.
This requirement of rapid feedback is consistent with galaxy scale simulations, such as \citet{2014MNRAS.445..581H}. 
%which require some early form of feedback in order to obtain the correct halo mass to stellar mass relationship \citep{2014MNRAS.445..581H}. 
Their simulations include explicit treatment of the multiphase ISM and stellar feedback, and
\citet{2014MNRAS.445..581H} found that those sources of feedback reproduce the observed halo and stellar mass relationship.
\citet{2015ApJ...804...18A} find that radiation pressure and efficient supernovae feedback together, is crucial, as removing any of these feedback sources affects
the star formation history.

The research which makes up the body of work in this dissertation was motivated, in large part, 
by the discrepancies between observations and the analytic models. 
Numerical simulations, as well, failed to account for the most recent insights into how stars form. 

\section{Isolating the effect of random thermal motions}
My initial simulations did not include the additional physics of magnetic fields, winds, jets or supernovae. 
This was an effort to understand how the system behaves in its simplest form.
%Motivated by disagreement between current analytic models, numerical simulations, and observations, we perform deep AMR simulations 
%of star formation in self-gravitating continuously driven hydrodynamic turbulence. 
%The purpose of proceeding without any additional physics, such as magnetic fields, stellar winds and jets, radiative feedback, supernovae, 
%is to determine how the zeroth-order system behaves.

The significance of my results is twofold---turbulence is, in fact, a dynamic variable, driven by adiabatic compression \citep{2012ApJ...750L..31R}, 
and the turbulence itself acts to slow the collapse. 
%Our results strongly support the basic premise of MC15: that turbulence is a dynamic variable 
%which is driven by adiabatic compression \citep{2012ApJ...750L..31R}, and that the turbulence
%in turn acts to slow the collapse. 
Furthermore, these outcomes are supported by observations made of massive star forming regions. 
As highlighted in this dissertation,  $\vt \propto r^p$ with $ p \sim 0.2-0.3$, and that at small radii or high density, $\vt$  
increases with increasing density, \citep{1997ApJ...476..730P}. 
We find these departures from Larson's law only in collapsing regions in our simulations. 
%We note, as did MC15, that observations of massive star forming regions also find $v_T \propto r^p$ 
%with $ p \sim 0.2-0.3$, and that at small radii or high density, $v_T$  
%increases with increasing density, as seen in observations of massive star forming
%regions \citep{1997ApJ...476..730P}. 
%We find these departures from Larson's law only in
%collapsing regions in our simulations. 

I show that two length scales emerge from the process of star formation, $r_*$ and $r_d$, and demonstrate that these length scales are clearly associated with physical effects. 
As mentioned in section \ref{sec:sf_review} analytic theories assumed that the solution shows a self-similarity, and thus would present a power law solution.
In fact, there are two power laws, which which change at $r_*(t)$. 
One power law inside of $r_*$ (but outside $r_d$) $|\ur|$ and  $\vt$ are both
$\propto r^{-1/2}$; and outside of $r_*$, $\vt \sim r^{p}$ (with $p\approx0.2$), while $|\ur|$ is on average about constant. 
It is worth emphasizing, that $r_*(t)$, the length scale at which the character of the solution changes, is time dependent. 
As the star grows in mass, the radius where the stars' gravity exceeds the gravity of the surrounding gas increases outwards away from the star, 
$r_*(t) \propto M_*^{2/3}(t)$.
The disk radius, $r_d$, also changes as a function of time as a result of the advection and 
transport of angular momentum from large scales to small scales (and vice versa).

I found that the density profile evolves to a fixed attractor, $\rho(r,t ) \rightarrow \rho(r)$. 

I also show that the acceleration due to the pressure gradient is comparable to that due to gravity at all $r>r_d$. 
As a result, the infall velocity is substantially smaller than the free fall velocity
even very close to the star or accretion disk. 
Inside $r_d$, rotational support takes over and as a result $\ur$ and $\vt$ both decrease.  

I found the development of rotationally supported disks at $r_d \sim 0.01$ pc.  
These disks have radii comparable to or slightly larger than disks seen around 
young stars in Taurus ($\sim 500 - 900\AU$) \citep{1999AJ....117.1490P} in which stellar feedback effects are minimal,
and where the undisturbed disks are larger than in more active star forming regions such as Orion, 
where the disk radii are $\sim 100\AU$ \citep{2011ARA&A..49...67W}.  
This is despite the fact that I do not include magnetohydrodynamic effects 
in my numerical computations; large scale magnetic fields may transfer angular 
momentum away from these disks, shrinking them.  

Like the disks modeled by \citet{2010ApJ...708.1585K},
my simulated disks are marginally gravitationally stable, 
suggesting that large scale gravitational torques are 
responsible for transport of material and angular momentum in our simulations; this may
also be true at early times in real protostellar disks.  

I have shown that the assumptions made by previous analytic collapse models
\citep{1977ApJ...214..488S,1992ApJ...396..631M,1997ApJ...476..750M,2003ApJ...585..850M}, 
are not fulfilled in my simulations. 
In my simulations, the collapsing regions do not start from a hydrostatic equilibrium, nor 
do they show any evidence of inside-out collapse. 
The gathering of material before collapse, i.e.,
before the central cusp in the density power law is formed, involves transonic bulk motions 
and supersonic random motions (see Figure \ref{fig:hydro_velocity_avg}). 
The accretion of mass starts at large scales ($r\sim 1\pc$) with large initial infall velocities. 
In addition, we find that $\vt$ 
scales differently in collapsing regions as opposed to the rest of the simulation box, 
whereas the turbulent collapse models \citep{1997ApJ...476..750M,2003ApJ...585..850M} assume 
that the scaling of $\vt$ with $r$ remains fixed.  

%Finally, we close with a brief discussion of how our results relate to turbulence regulated theories of star formation. Here we find several points of disagreement. 
%First, we find that the star 
%particles accrete continuously from the surrounding large scale turbulent flow; there is 
%no hydrostatic ``core'' that is cut-off from the turbulent medium. Second, 
%the density distribution does not remain log-normal, but rather develops a power law tail that 
%is directly related to the density profile \citep{2011ApJ...727L..20K,2015ApJ...800...49L}.  
%Third, the fact that the density profile approaches an attractor solution that scales like 
%$r^{-3/2}$ for $r<r_*$ and $u_r$ scales with the Keplerian velocity guarantees that $\dot{M}$ is 
%constant with radius and $\dot{M}_*\propto t$ and hence a non-linear star formation efficiency, 
%i.e., $M_* \propto t^2$ results. This is in contrast with turbulence regulated theories of star 
%formation that predict a constant star formation rate, i.e., $\dot{M}_* = {\rm const}$ and, 
%hence, a linear star formation efficiency $M_* \propto t$. 

\section{The effect of protostellar Jets}
%We performed simulations of turbulent, self-gravitating gas including star particle formation and protostellar jets. 
%Starteing with uniform density in a box with length $16$ parsecs on a side, we drove turbulence until we reached a statistically steady state.
%At that point, the density was no longer uniform.
%We then turned on gravity and star formation.
%We used AMR to follow collapsing regions down to an effective resolution of $32 \, K^3$ which gave us a $\Delta x$ of $100 {\rm AU}$ at the finest level of refinement.
%The spherical average profile of gas around the protostar does not appear to change in the case with jets.

We observed that the inclusion of protostellar jets does not affect the general dynamics of accreting gas.
In particular, we saw $M_*(t) \propto f^2 (t-t_*)^2$ where $f = 1 - f_{\rm jet}$ is the fraction of mass accreted onto the protostar and $f_{\rm jet}$ is the fraction ejected by the jet.
We find that this mass ejection accounts for 75\% of the effect of jets on the star formation rate in our simulations.
This appears to be the case in similar simulations performed by other groups (e.g. see Figure \ref{fig:jet_fed_comp}), but we find suggestions that this may be altered if MHD is included.

As found previously, in the case without jets \citep{2017MNRAS.465.1316M}, 
the spherical average profile of gas around the protostar follows the analytic model of MC15 and does not appear to change in the case with jets.
%As we have found previously in the case without jets \citep{2017MNRAS.465.1316M}, the spherical average profile of gas around the protostar follows the analytic model of MC15 and does not seem to change in the case with jets.  
In particular, the run of density finds an attractor solution prior to star formation and remains on that solution even after jets begin to blow out cavities in the surrounding medium.
The behavior of the infall and rotational velocities is similar regardless of whether jets are included or not.
The profile of the random velocities is also similar, once the jet bi-cone is removed.  Finally, the mass accretion rates are similar in the jet and no-jet cases.

Of particular note is that the collapse is outside-in  \citep{2017MNRAS.465.1316M}, and holds for both the jet and no jet simulations.
The average jet momentum per stellar mass does increase over time, though this is to be expected as the stars continue to accrete mass.
Our runs were not long enough for the stars to completely consume the surrounding gas and thus, for the jets to begin to be shut off.
We find that jets do drive turbulence in the surrounding gas, but is confined to small scales of roughly a parsec.

\section{Future Work}
While the results presented in this dissertation move the story of star formation a little further along, 
there is still much we do not understand and further work is needed. 

For instance, this theory and the results of this dissertation, has not yet looked at the effects of magnetic fields. 
There are several papers \citep{2009ApJ...704..891L,2015ApJ...808...48B,2017ApJ...838...40M} , which suggest that magnetic fields have an effect on the rate of star formation. 
Whether that contribution modifies the $M_*(t) \propto t^2$ profile we've seen or the normalization, like protostellar jets, has yet to be determined. 

As mentioned above, my simulations also neglect strong physical effects at late times. 
For example, the absence of radiation is not physically accurate for sink particles that grow beyond $\sim 4 \Msun$.
We direct the reader to section \ref{subsec:missing_physics} for our argument about $M < 4 \Msun$. 
The upshot was that we expect that the effects of radiation pressure are not particularly significant in the situations we report; 
the run of density and infall velocity, and hence the $M_*(t)\sim t^2$ scaling should not be affected, at least up to the times we are reporting on. 
This estimate neglects the effect of radiative or ionization heating which is an important feedback mechanism. 
In addition, this our argument in \ref{subsec:missing_physics} indicates that radiation dynamics must have a significant effect when $M_* \ge 4 \Msun$. 

%as the mass of stars grows the effects of radiation from these massive stars will have a strong impact on the surrounding medium. 

Radiative feedback will also affect both the IMF and, for massive enough stars, the dynamics of the collapse at late times (after massive stars have formed). 

%To estimate the effects of radiation, we compare the force from the Reynold stress $F_T=4\pi r^2 \rho \vt^2$, to the radiation force $L/c$. 
%From Figure \ref{fig:hydro_velocity_avg}, the (averaged over many stars) $\vt$ is slightly in excess of $1\kms$ at $r=0.01\pc$, 
%while from any of the density figures the density is $\rho\approx5\times10^{-18} {\rm g/cm}^3$. 
%The force from Reynold stress is then $F\approx4\times 10^{26}$ dynes. 
%The luminosity of a 4 solar mass star on the zero age main sequence is $L\approx 2\times10^{36}\ergs$ \citep{1992A&AS...96..269S}, 
%so the radiation force $L/c \approx 3\times10^{25}\,{\rm dynes}$, about a 10\% effect. 
%The force from Reynolds stress increases outward, see Figure \ref{fig:hydro_dPdr_avg}, so this statement holds at larger radii as well. 
%
%Thus we expect that the effects of radiation pressure are not particularly significant in the situations we report; 
%the run of density and infall velocity, and hence the $M_*(t)\sim t^2$ scaling should not be affected, at least up to the times we are reporting on. 
%We note, however, that this estimate neglects the effect of radiative or ionization heating which is an important feedback mechanism.

Simulations including radiative feedback support this simple analysis. 
Figure 15 of  \citet{2014MNRAS.439.3420M} shows that in their simulations, 
which include feedback from both protostellar outflows and radiation (as well as magnetic fields), 
the stellar mass increases as the square of the time, up to masses of ~4.5 solar masses. 
Earlier work by the Berkeley group found similar results, forming stars with 10 solar masses, 
with $M_*(t)~ t^2$ even for such massive stars, see Figure 13 of \citet{2012ApJ...754...71K}. 
Their simulations included radiative effects, but no proto-stellar winds. 

Among the myriad of contributions that can be made, looking at the chemistry of these regions, 
creating predictions about the species, the type and amount of emission and absorption of radiation that may be seen. 
The generation of synthetic observations from these simulations would allow more accurate testing by the latest generation of telescopes.

\end{document}

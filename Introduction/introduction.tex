\documentclass[../dissertation.tex]{subfiles}
%\setlength{\epigraphwidth}{.7\textwidth}

\begin{document}

\chapter{Introduction}
\label{ch:intro}

\singlespace
\epigraph{``\emph{Everything starts somewhere, though many physicists disagree. But people have always been dimly aware of the problem with the start of things. They wonder how the snowplough driver gets to work, or how the makers of dictionaries look up the spelling of words}"}{--- \textit{Hogfather by Terry Pratchett}}

\dblspace

%Start close to home: our Sun provides all the energy that we have/use on Earth.
Stars are arguably the fundamental building blocks of the Universe, 
and play a crucial role in the existance of life on Earth, and in phenomena seen across the Universe.
All of the elements that make up the planets of our solar system, including oxygen, silicon, and iron, as well as elements found in living organisms, like carbon and nitrogen were created by stars. 

%\subsection{The energy that drives life on Earth}
The energy that fuels life and our modern society comes exclusively from stars. 
The oil that we burn, the wind we capture via turbines, 
and the radiation we capture via solar panels, all have an origin in the nuclear fusion reactions in the center of our Sun. 
The one exception is the fuel used in nuclear reactors, and that elements used as fuel were created via the death of stars older than our Sun.

%\subsection{What makes up a star?}
As the Sun is our nearest star, it has played a crucial role in our understanding of astrophysical information. 
The vast majority of newborn stars in our Milky Way Galaxy are made of $\sim 71 \%$ Hydrogen, $\sim 27 \%$ Helium, and a smattering of heavier elements. 


\section{How do we observe stars}





\section{Where Are Stars Born?}


The journey of star formation begins in the dust and cold clouds of gas called nebulae. Nebulae are many light-years across and have the potential to create thousands of stars about the size of our sun. Nebulae contain essential ingredients for the production of stars, such as hydrogen and helium. The Small Magellanic Clouds are mottled with star forming regions. These regions or clusters-- en masse-- make up an irregular galaxy, a kind of galaxy more commonly found in the Early Universe. These cold clouds begin to collapse under their own gravity, becoming smaller, more dense, and more hot. The collapsing cloud becomes dense enough to initiate a thermonuclear reaction. At around a temperature of 10 million degrees Celsius, a nascent star emerges out of the hot dense clumps of material. Observations tell us that most of these young stars form near or at the center of the collapsing cloud, lying in what is an emerging flat disc. To paraphrase author Terry Pratchett, this is truly ``the start of things'' astronomical. Planets may form around this emerging star.


With our eyes turned to the cosmos, we must now ask, where do we see new or young stars being born? 
What are the physical conditions that lend themselves to the formation of stars, 
and is there a range of environments in which stars may form?
To investigate, let us first turn our attention to the space between stars;  
The interstellar medium (ISM) is the rather diverse environment of atomic gas, molecular gas, and radiation. 
It fills a crucial role because of its intermediate role between the stellar scale and the galactic scale. 
Our Sun, and indeed every star interacts with the ISM and this interaction leads to a diverse set of observable structures. 
In addition, our Sun is currently traveling through our Local Interstellar Cloud. 
While there is a myriad of different environments and structures that form the ISM, we have only seen stars form in molecular, not in atomic gas. 
These observations place young stellar objects in clouds of molecular gas and dust that can be 
as large as 100 parsecs in size, or roughly the height of the galactic disk of our Milky Way. 
These massive clouds of gas are referred to as Giant Molecular Clouds (GMCs). 
But how do we know that stars only form in these regions of cold gas?

Before we present evidence for this, it is quite reasonable to believe that stars would result in these clouds; 
in order to accrete the large amount of matter in order to form a star, 
the infant proto-stars would need a large resevoir of gas and dust. 
This gas and dust must also be cold enough so that thermal motions would not be enough to prevent the gas from collapsing under its own self-gravity. 
In addition, at lower temperature, not only can atoms recombine, but molecules may form as well. 
Thus, we can identify regions of cold dense gas via specific molecules, referred to as tracers, which dissociate at higher temperatures. 
One of the more popular tracers used for this purpose is Carbon Monoxide. 

With this information we can now identify locations in our galaxy (finding cold gas in extragalactic locations is not really feasible)
that have cold gas.

Young massive stars and O stars especially, emit a form of radiation referred to as free-free emission. 
Free-free emission is radiation which is created by charged particles that are, as the name implies, free; i.e. not part of a molecule or atom. 
In order to have free floating electrons, one requires a significant temperature to strip and ionize the surrounding atoms. 
These temperatures are around 10,000 Kelvin or higher, in order to strip electrons off of hydrogen atoms.
As the reader may infer, this necessitates that any CO no longer exists, as the molecule would be destroyed in the face of that extreme radiation. 

Observations of the galactic plane looking at free-free emission and CO emission shows that the two tracers map over each other

\section{What impact do stars have on surrounding regions?}

Stars are arguably the fundamental building blocks of the universe. They are the cauldrons where Nature manufactures heavy elements. They are the constituents of galaxies. They give rise to planetary systems. 

\section{Star Formation Efficiency (SFE)}

They are responsible for the creation of most of the elements, including  oxygen, silicon, and iron, that make up
terrestrial planets like the Earth and Venus, as well as these and other elements found in
living organisms, like carbon and nitrogen. 
Stars are also responsible for regulating the
amount of gas in galaxies, as well as for the ejection of heavy elements into the intergalactic
medium. 
In doing so, stars regulate the total fraction of gas that ends up in stars and planets. 
That is, they regulate the efficiency with which gas is converted into stars. 
Observations of galaxies have shown that, at most, only one quarter of the gas available to a given galaxy is
converted into stars; for most galaxies, the fraction is much smaller. 
Why this should be, and how the fraction is determined, is currently one of the most important topics in astrophysics.


The journey of star formation is the collapse of a cloud of gas into a dense sphere that begins thermonuclear fusion.
A measure of how effective nature is at converting molecular gas into stars is the star formation effeciency (SFE):
%
\be
{\rm SFE} = \frac{M_*}{M_{\rm gas}},
\ee
%
where $M_*$ is the total mass in stars and $M_{\rm gas}$ is the total mass in gas.

If we imagine an isolated cloud of gas with constant density, $\rho(r) \rightarrow \rho$, we can derive the amount of time required for the cloud of gas to collapse to a single point.
This time is the free-fall time:
%
\be
t_{\rm ff} = \sqrt{\frac{3 \pi}{32 G \rho}},
\ee
%
where $ {\rm G} = 6.67 \times 10^{-8} $ is Newton's constant.
In this simple model, if all of the gas collapses to a point in a free-fall time, then the SFE would 100\%.
The question we need to answer is does this zeroth-order model hold observationally, and if not, what physics dominates the dynamics?

\section{Kennicutt-Schmidt law}
The star formation time on galactic scales is long when compared to the dynamical time. 
\citet{1998ApJ...498..541K} expressed this in the form
%
\begin{equation}\label{eq:intro_KS_law}
\dot{\Sigma}_* = \epsilon\Sigma_g \tau_{\rm DYN}^{-1}
\end{equation}
%
where $\dot{\Sigma}_*$ is the star formation rate per unit area, $\Sigma_g$ is the 
gas surface density, $\tau_{\rm DYN}$ is the local dynamical time, and $\epsilon = 0.017$ 
is the efficiency factor.  

In our rather naive model, if the gas self-gravity dominates the dynamics, $\epsilon \sim 1$, so the low efficiency of star formation is surprising. 
More recent work has refined this and similar relations in regard to its dependence on molecular gas \citep{2008AJ....136.2846B} and by taking into account the error distributions of both $\dot\Sigma_*$ and $\Sigma_g$ \citep{2013MNRAS.430..288S}, but the best current estimates of the efficiency of star formation on galactic scales remains low. 

Whether this low efficiency applies to scales comparable to giant molecular clouds, with radii of
order $100\pc$, is debated in the literature.
\citet{2010ApJ...723.1019H}, \citet{2010ApJ...724..687L}, \citet{2010ApJS..188..313W}, and \citet{2011ApJ...729..133M} 
find efficiencies a factor of ten or more larger, while \citet{2007ApJ...654..304K} and \citet{2012ApJ...745...69K} 
find $\epsilon\approx0.01$.
On these small scales, observations also suggest that the efficiency is not universal, but instead 
varies over two to three orders of magnitude (e.g.  
\citealt{1988ApJ...334L..51M,2016arXiv160805415L}).%2015ApJ...804...18A}).  

There are a number of explanations for the low star formation rate, on 
either small or large scales (although they may not be necessary for the former!). 
These include turbulent pressure support \citep{1992ApJ...396..631M}, support from magnetic fields
\citep{1966MNRAS.132..359S,1976ApJ...207..141M}, and stellar feedback (e.g.
\citealt{1986ApJ...303...39D}).  

A number of explanations for this low star formation rate and scatter in the efficiency, on either galactic or GMC scales, have been put forth.
On large scales, the leading candidate is stellar feedback, e.g. \citet{1986ApJ...303...39D}, in which supernovae limit the amount of dense gas. 
On small scales, these include turbulent pressure support \citep{1992ApJ...396..631M} and support from magnetic fields \citep{1966MNRAS.132..359S,1976ApJ...207..141M}. 
Numerical experiments investigating turbulence and magnetic fields suggest that, while magnetic support found in MHD simulations can slow the rate of star formation 
compared to hydrodynamics simulations, neither turbulence nor magnetic support is sufficient to limit the small scale star formation rate to 1-2\% per free fall time
\citep{2010ApJ...709...27W,2011MNRAS.410L...8C,2011ApJ...730...40P,2012ApJ...754...71K,2014MNRAS.439.3420M,2015ApJ...808...48B,2017ApJ...838...40M}.

\section{Review of Analytic Theories of Star Formation} \label{sec:sf_review}
I will begin with a refresher on the state of analytic theories of star formation, and what each individual step has indicated for our understanding of this process.
Beginning in the late 1960's, star formation theory revolved around hydrostatic equilibrium (HSE) supporting a cloud of gas and dust via thermal pressure support. 
The thought process is that eventually, the density/mass of the cloud becomes too great for the thermal pressure support, and the cloud collapses.

\subsection{\citet{1969MNRAS.145..271L} and \citet{1969MNRAS.144..425P}}
\citet{1969MNRAS.145..271L} and \citet{1969MNRAS.144..425P} first put forth the idea that the collapse of this core would have a self similar solution.

\citet{1969MNRAS.145..271L}
Talk about Larson 1969 \& Penston 1969

\subsection{Shu 1977} \label{subsec:Shu_review}
\citet{1977ApJ...214..488S} found that the solution described by \citet{1969MNRAS.145..271L} and \citet{1969MNRAS.144..425P} was physically artificial prior to core formation.
One of the issues with the model was the need to have the flow at large radii be directed inwards at velocity ${\rm v} \rightarrow \infty$ as $r \rightarrow \infty$. 
This property is clearly non-physical, and it is not apparent that as the flow could stably transition from supersonic to subsonic, and then to rest. 
In addition, only specific initial and boundary conditions could lead to the Larson-Penston flow.
However, \citet{1977ApJ...214..488S} found that the flow following core formation dows exhibit self-similar properties described in \citet{1969MNRAS.145..271L}.
Thus, to rectify this issue with the boundary condition, \citet{1977ApJ...214..488S} explicitly assumed that the coud is hydrostatic for radii larger than $r = c_s t$. 
At $t=0$ a perturbation causes the central region to collapse. 
For regions where $r < c_s t$, the layers find that their pressure support inside of them has disappeared and begin to fall inwards. 
This collapse expands outwards in time, leading to the phrase ``inside-out collapse''. 
Following this solution, \citet{1977ApJ...214..488S} estimated the accretion rate onto 
stars by assuming that stars form from hydrostatic cores supported by thermal gas pressure. 
The accretion rate in his model was independent of time, given by $\dot{M} = m_0c_s^3/G$, where $c_s = (k_b T / \mu)^{1/2}$  is the sound
speed in molecular gas, and $m_0 = 0.975$. 
\citet{1977ApJ...214..488S} predicted a maximum accretion rate of $\sim 2\times10^{-6} M_\odot\,\yr^{-1}$, which is too small to explain 
the origin of massive ($M_*\sim50-100M_\odot$) O stars, which have lifetimes $\lesssim 4\times10^6\yrs$.

\subsection{Turbulent Pressure Support}
\citet{1992ApJ...396..631M} overcame this difficulty with slow accretion rates by adopting 
the turbulent speed in lieu of the sound speed (see also \citet{1997ApJ...476..750M} and
\citet{2003ApJ...585..850M}). 
In doing so they were able to replace the slower signal speed of sound with the faster turbulent speed. 
However, they continued to assume the initial condition was that of a hydrostatic core that is supported by
turbulent pressure. They also assumed that the turbulence is static and unaffected by the collapse, i.e. $\vt (r) \rightarrow \vt$.  

Collectively, these models, \citep{1977ApJ...214..488S,1992ApJ...396..631M,1997ApJ...476..750M,2003ApJ...585..850M}, 
are referred to as inside-out collapse models; the collapse starts at small radii (formally at $r=0$ in the analytic models) 
and works its way outward, at the assumed propagation speed ($c_s$ or $\vt (r)$). 
At any given time, the infall velocity and mass accretion rate both decrease with increasing radius $r$. 
The analytic models assume the existence of a self-similarity variable $x = r/vt$, where $v=c_s$ in 
\citet{1977ApJ...214..488S} or the turbulent velocity $\vt (r)$ in \citet{1992ApJ...396..631M,1997ApJ...476..750M,2003ApJ...585..850M}.  
These models predict velocity and mass accretion profiles very different than those seen 
in the simulations of \citet{2015ApJ...800...49L}.

\section{Current Numerical simulations and Observations} \label{current_obs}
Numerical experiments investigating the effects of turbulent pressure and magnetic fields suggest that neither turbulence nor magnetic support is sufficient 
to reduce the rate of star formation to $\epsilon\approx 0.02$ on small scales
\citep{2010ApJ...709...27W,2011MNRAS.410L...8C,2011ApJ...730...40P,2012ApJ...754...71K,2014MNRAS.439.3420M}. 
Feedback from radiative effects and protostellar jets and winds may be able to explain the low star formation rate, 
but the impact of these forms of stellar feedback remains uncertain despite recent progress
\citep{2010ApJ...709...27W,2014MNRAS.439.3420M,2015MNRAS.450.4035F}.

Until very recently, galaxy-scale or larger (cosmological) simulations 
were not able to reproduce the Kennicutt-Schmidt relation. 
Nor did the cosmological runs reproduce correctly the mass of stars in galaxies of a given 
halo mass, despite including supernova and other forms of feedback, 
e.g., \citet{2010MNRAS.404.1111G,2010Natur.463..203G,2011MNRAS.410.2625P}.
To overcome this low resolution driven problem, \citet{2011MNRAS.417..950H,2012MNRAS.421.3522H}
performed high resolution (few parsec spatial, few hundred solar mass particle masses) simulations 
of isolated galaxies, modeling both radiative and supernovae feedback (among other forms). 
They recovered the Kennicutt-Schmidt relation, a result that they 
showed was independent of the small-scale star formation
law that they employed. 
The simulations in the second paper also generated galaxy 
scale outflows or winds, removing gas from the disk, thus making it unavailable for star 
formation. 
When the feedback was turned off, the star formation rate soared, demonstrating that
in the simulations at least, feedback was crucial to explaining the Kennicutt-Schmidt relation, and
the outflows. 
Simulations including supernovae but lacking the radiative component of the feedback
did not exhibit strong winds and so overproduced stars.

Cosmological simulations employing unresolved (or ``sub-grid") models for both 
radiative and supernovae feedback are now able to reproduce the halo-mass/stellar mass relation (e.g.,
\citealt{2013MNRAS.434.3142A,2014MNRAS.445..581H,2015ApJ...804...18A}). Again, 
these simulations {\em require} 
stellar feedback to drive the winds that remove gas from the disk, so as to leave the observed mass of stars behind. 


\subsection{Lee et al. 2015 Numerical Simulation} \label{Lee15_model}

\citet{2015ApJ...800...49L} showed that, in simulations with no feedback, the star formation efficiency on
parsec scales is not constant in time.
This is in contrast to previous work, which had implicitly assumed that the star formation rate on small scales was constant.
In particular, many authors have assumed that the star formation rate in their simulations of a GMC (or smaller cloud or part of a cloud) was given by equation (\ref{eq:intro_KS_law}), where $\epsilon$ was assumed to be constant.
\citet{2015ApJ...800...49L} showed that $\epsilon \propto t$, which implies that $M_* \propto t^2$, where $M_*$ is the total stellar mass.

\citet{2015ApJ...800...49L} emphasized that the star formation efficiency on 
parsec scales is nonlinear 
in time, i.e., $\epsilon \propto t \rightarrow M_* \propto t^2$, on small scales, where $M_*$ is the total stellar mass. 
%Magnetic fields slowed the initial star formation rate somewhat, but did not change the $M_*(t)\sim t^2$ scaling. 
Using a detailed numerical simulation, they showed that this nonlinear star formation 
rate is driven by the properties of collapsing regions.
In particular, they demonstrated that the turbulent velocity near or in collapsing regions follows 
different scaling relations than does turbulence in the global environment, which follows
Larson's law, $\vt (r)\sim r^{1/2}$ \citep{1981MNRAS.194..809L}. 
They also showed that the density PDF 
is not log-normal, but rather develops a power law to high density.
This latter result was hinted at by \citet{2000ApJ...535..869K} and shown convincingly, 
as well as explained, by \citet{2011ApJ...727L..20K}. 

The increasing rate of star formation found by \citet{2015ApJ...800...49L} is important 
in that it may provide an explanation for the observed range in star formation rates on 
small scales.  It suggests that the star formation rates on small scales 
vary in part because of the age of the star forming region; slow star forming 
regions, with very low instantaneous efficiencies, will ramp up their stellar production over time. If this result
can be firmly established, it will highlight the need for a form of very rapid 
feedback. In particular, since the dynamical time in massive star forming regions 
is much smaller than the time delay of $\sim 4{\,\rm Myrs}$ between the start of star 
formation and the first supernovae, rapid star formation on small scales would 
have to be halted by some form of feedback other than supernovae. 

The simulations of  \citet{2015ApJ...800...49L} explicate the link between the rate of star formation with the gravitational collapse of high density regions, which  is  an analytically well studied problem. 

\subsection{Murray and Chang 2015} \label{MC15_model}
%Motivated by this, \citet[][hereafter MC15]{2015ApJ...804...44M} developed a new 1-D model of spherical collapse that treats the turbulent velocity, $v_T$, as a dynamical variable.
%%In essence, they extended the classic analysis of \citet{1977ApJ...214..488S} and \citet{1992ApJ...396..631M}; see also \citet{1997ApJ...476..750M,2003ApJ...585..850M}.
%%
%Following earlier work \citep{1977ApJ...214..488S,1992ApJ...396..631M,1997ApJ...476..750M}, MC15 reduced the fluid equations by assuming spherical symmetry, but did not assume a fixed equation of state.

Motivated by this discrepancy between observation and the current analytical models,
\citet{2015ApJ...804...44M}, hereafter MC15, developed a 1-D model of
spherical collapse that treats the turbulent velocity, $\vt$, as a dynamical variable and
does not assume that the initial condition is a hydrostatically supported region. 
They used the results of \citet{2012ApJ...750L..31R} on compressible turbulence; the evolution 
of the turbulent velocity in a collapsing (or expanding) region is described well by the following equation:
%
\be
\frac{\partial \vt}{\partial t} + \ur \frac{\partial \vt}{\partial r} 
+ \left( 1 + \eta \frac{\vt}{\ur} \right) \frac{\vt \ur}{r} = 0
\label{eq:hydro_Robertson}
\ee
%
The first two terms are the Lagrangian derivative, and $\ur$ is the radial infall velocity. The first term in the brackets 
describes the turbulent driving produced by the infall, while the second 
is the standard expression for the turbulent decay rate;  $\eta$ is a dimensionless constant of order unity.

MC15 used this in place of an energy equation. Together with the
equations for mass continuity and momentum, equation (\ref{eq:hydro_Robertson}) gives a
closed set of equations that can be solved in spherical symmetry
numerically. In addition, they were able to analytically show that the
results of their calculations gave density and velocity profiles that
appear to be in line with both recent numerical calculations
\citep{2015ApJ...800...49L} and observations (e.g.,
\citealt{1995ApJ...446..665C,1997ApJ...476..730P}).

To summarize, MC15's major results were:
\begin{itemize}
\item The gravity of the newly formed star introduces a physical scale into the problem, which MC15 called the stellar sphere of influence, $r_*$. This is an idea familiar from galactic dynamics. 
The radius $r_*$ is where the local dynamics transitions from being 
dominated by the mass of the gas to being dominated by the mass of the star. 
As a result, the character of the solution, in particular that of the velocity, differs dramatically between $r<r_*$ and $r>r_*$. 
The existence of this physical scale modifies the form of the self-similarity on which inside-out theories rely.

\item The small scale density profile is an attractor solution.  MC15 showed numerically and argued analytically that at small scales, the density profile is an attractor solution. 
In particular, MC15 showed the density profile asymptotes to: 
%
\be
\rho(r,t)=
\begin{dcases}
\rho(r_0)\left({\frac{r}{r_0}}\right)^{-3/2}, & r<r_*(t)\\
\rho(r_0,t)\left({\frac{r}{r_0}}\right)^{-k_\rho}, \ k_\rho\approx1.6-1.8 & r>r_*(t).
\end{dcases}
\label{eq:hydro_density}
\ee
%
where $r_0$ is some fiducial radius. 
\item The existence of $r_*$ implies that the infall and turbulent velocities have different scaling for $r<r_*$ and $r>r_*$.  In particular, MC15 showed
%
\be
u_r(r,t), v_T(r,t) \propto
\begin{dcases}
\sqrt{\frac{GM_*(t)}{r}} \sim r^{-1/2} & r<r_*(t)\\
\sqrt{\frac{GM(r,t)}{r}} \sim r^{0.2} & r>r_*(t),
\end{dcases}
\label{eq:hydro_infall_behavior}
\ee
%
Thus the scaling of the turbulent velocity differs from that predicted by Larson's law ($\propto r^{1/2}$) inside the sphere of influence. In other words, the turbulent velocity in  massive star forming regions will deviate from Larson's law, which has long been observed, but without theoretical explanation. 
\item The stellar mass increases quadratically with time.  This result  arises naturally from the attractor solution nature of the density profile at small $r$, Equation (\ref{eq:hydro_density}), and the scaling with Keplerian velocity for the turbulent and infall velocities at small $r$, Equation (\ref{eq:hydro_infall_behavior}).

The mass accretion rate: 
%
\be
\dot M(r,t)=
\begin{dcases}
4\pi R^2\rho(R)u_r(r,t), \sim t\,r^{0} & r<r_*\\
4\pi R^2\rho(R)u_r(r,t) \sim t^0\,r^{0.2} & r>r_*.
\end{dcases}
\label{eq:hydro_Mdot_behavior}
\ee
%
\end{itemize}

MC15's predictions for $r<r_*$ could not be checked using the simulations
of \citet{2015ApJ...800...49L} as those fixed grid simulations were
too coarse.  

This dissertation describes my contributions to various parts of our current understanding of the dynamics of star formation. 
This chapter gives a brief overview of the theory behind star formation and introduces descrepancies in our current understanding. 
Ch. \ref{ch:hydro} discusses the initial set up for the majority of my simulations, and the dynamics of gravitational collapse with no feedback or other delaying aspects.
Ch. \ref{ch:jet} discusses the effects that protostellar and jet feedback have on the stellar mass accretion rate for a cluster of stars.
Ch. \ref{ch:mhd} discusses the dynamical effect of magnetic fields. %TODO Confirm if removing.
Finally, in Ch. ~\ref{ch:conclusions}, I summarize the results obtained in this dissertation succinctly.

\end{document}

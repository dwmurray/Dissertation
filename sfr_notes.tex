The optical spectra of many galaxies contains conspicuous emission lines. The spectra of many hot stars



Astrophysics is based upon the analysis and interpretation of the way in which matter interacts with light, and electromagnetic radiation in general. 
Electrons in the excited state if left alone for a memnt of time will return to the ground state. The act of doing so emits a packet of waves referred to as a photon. 
The key for astronomers is that each transition will emit a photon of a specific wavelength and only that wavelength. 
This consistancy is the key to our understanding of the elements, and state of those elements associated with whatever phenomena we are looking at. 
In the case of star formation, astronomers look for the emission from hydrogen transitioning from the first excited state to the ground state. 
The emission that is released has a wavelength of 21cm, and is referred to as Lyman Alpha or 21cm emission. 
Lyman alpha emission is incredibly important in astronomy, due to several factors: 
The vast majority of matter in the universe is Hydrogen, and our galaxy is invisible to 21cm emission. 
That is to say, the 21cm emission from the source location can travel through our galaxy to our telescopes, with only a small possibility of being absorped. 
Another important result is that the width of the 21cm emission line can provide us with information regarding the thermal velocity of the gas that is emitting. 

Absorption is as important as emission in astronomy. 
If a region of space has continuous emission, but has absorption lines, this as well provides us with information 
regarding the chemical composition, velocity and _________ of the region of interest. 
Emission requires the atoms to be excited in some manner, either by thermal collisions or another process.
Absorption requires radiation from some source to be shined upon the atoms, thus exciting those atoms. 



The intentsity of 21-cm photons received from a certain direction depends only on the particle column density o femitting hydrogen atoms.
 A column density is defined as the total number of atoms in a long thin cylinder that extends indefinitely far along the line of sight. 
Thus, it is expressed as a number of atoms per cross-sectional area. 

Observations of hot young stars in our Galaxy show that they are mostly concentrated in the spiral arms. 
The average velocity of these young stars measured in the same direction as the 21-cm emission lines is about equal to that seen in the 21-cm lines. 
Thus, we see that the interstellar hydrogen gas moves with the young stellar objects, ksuggesting that the gas is concentrated in the regions in which bright young stars are found. 
